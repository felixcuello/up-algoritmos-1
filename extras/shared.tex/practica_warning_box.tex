\begin{tcolorbox}[colback=yellow,colframe=red!75!black,arc=0pt,outer arc=0pt]
  \textbf{ATENCIÓN CON LAS PRÁCTICAS} \\

La práctica es un componente fundamental del aprendizaje en programación. Hacer estas prácticas te van a dar la
oportunidad de poder aplicar lo aprendido en la teoría, darte cuenta si fuiste entendiendo los conceptos que se fueron
dando en clase.

\begin{itemize}
\item \textbf{Leé MUY bien el enunciado}: Los problemas están pensados para que puedas resolverlos aplicando sólo lo que
viste en la clase hasta ese momento. No hay trucos ocultos.

\item \textbf{No busques soluciones óptimas inmediatamente}: Primero tratá de pensar cómo harías el problema como se te
  ocurra, y luego si puede ser optimizado.

\item \textbf{Dedicale tiempo a cada ejercicio}: No te preocupes si no se te ocurre la solución inmediatamente. Es
bastante común que las soluciones no salgan rápido. A veces es necesario que vuelvas a leer la teória y fijarte si
releyendo la teoría se te ocurre algo nuevo.

\item \textbf{¡No busques soluciones en internet!}: El objetivo es que \textit{aprendas a resolver problemas}, no que
  copies soluciones.

\item \textbf{¿Qué hago si no me sale?}: Dale tiempo a los problemas, a veces es necesario dejarlos un tiempo y volver a
  verlos luego.

\item \textbf{¡Volví al problema luego de un tiempo y no me sale!}: Si volviste al problema y no se te ocurre nada, es
momento de leer las pistas (si hay), y si aún así no se te ocurre nada, podés pasar a ver la solución.

\item \textbf{Tratá de entender la solución}: El objetivo de las soluciones es que puedas entender una forma de resolver
  el ejercicio (puede no ser la única forma).

\item \textbf{¿Y si no entiendo la solución?}: Si no entendiste la solución, anotá las dudas y preguntá a los docentes
de la cátedra. \textbf{¡No te quedes con dudas!}

\item \textbf{¿Por qué estudiamos esto si un LLM como ChatGPT lo puede hacer más rápido?}: Si bien es cierto que un LLM
puede hacer esto, en un futuro vas a necesitar controlar, corroborar y entender el código que genera para poder saber
si la solución entregada es la correcta.

\textbf{¡Suerte en la práctica!}
\end{itemize}
\end{tcolorbox}
