% \documentclass[10pt, twocolumn, a4paper]{article}
\documentclass[12pt,a4paper]{article}

\usepackage[utf8]{inputenc}                                             % To handle UTF-8 input
\usepackage[T1]{fontenc}                                                % Better font encoding that supports accents
\usepackage[backend=biber, style=ieee]{biblatex}                        % To include the bibliography
\usepackage[left=2cm, right=2cm, top=2.5cm, bottom=4cm]{geometry}     % To set the margins
\usepackage[noend]{algpseudocode}
\usepackage[table]{xcolor}                                              % For coloring cells

\usepackage{algorithm}                                                  % To include algorithms
\usepackage{amsfonts}                                                   % To include math fonts:ToggleTerm direction=float
\usepackage{amsmath}                                                    % To include Mathematic symbols
\usepackage{authblk}                                                    % To format author affiliations
\usepackage{caption}                                                    % For caption spacing
\usepackage{enumitem}                                                   % To customize lists (items like i, ii, iii, iv)
\usepackage{float}                                                      % To place figures where you want them
\usepackage{fancyhdr}                                                   % To customize headers and footers
\usepackage{graphicx}                                                   % To include images
\usepackage{hyperref}                                                   % To include hyperlinks
\usepackage{lipsum}                                                     % TODO: remove this
\usepackage{pgfplots}                                                   % To plot functions
\usepackage{listings}                                                   % To include code
\usepackage{tabularx}                                                   % For equal-width columns
\usepackage{tcolorbox}                                                  % To make colored boxes
\usepackage{tikz}                                                       % To draw graphs
\usepackage{titlesec}                                                   % To format section titles
\usepackage{xcolor}                                                     % To define colors

\usetikzlibrary{graphs,graphs.standard, arrows.meta}
\usetikzlibrary{positioning}

\addbibresource{./references.bib}

%%%%%%%%%%%%%%%%%%%%%%%%%%%%%%%%%%%%%%%%%%%%%%%%%%%%%%%%%%%%%%%%%%%%%%%%%%%%%%%%
% Estilo de código
%%%%%%%%%%%%%%%%%%%%%%%%%%%%%%%%%%%%%%%%%%%%%%%%%%%%%%%%%%%%%%%%%%%%%%%%%%%%%%%%
\definecolor{up-blue}{RGB}{0,102,254}
\definecolor{up-violet}{RGB}{100,30,180}
\definecolor{up-gray}{RGB}{102,102,102}
\definecolor{up-red}{RGB}{152,102,52}
\definecolor{up-comment}{RGB}{153,153,153}
\definecolor{code-bg}{RGB}{248,248,248}
\definecolor{code-border}{RGB}{104,104,104}

\lstdefinestyle{up-code}{
    backgroundcolor=\color{code-bg},
    commentstyle=\color{up-comment}\itshape\footnotesize,
    keywordstyle=\color{up-blue}\bfseries,
    numberstyle=\tiny\color{up-red},
    stringstyle=\color{up-violet},
    basicstyle=\ttfamily\footnotesize,
    breakatwhitespace=false,
    breaklines=true,
    captionpos=b,
    keepspaces=true,
    numbers=left,
    numbersep=5pt,
    showspaces=false,
    showstringspaces=false,
    showtabs=false,
    tabsize=2,
    frame=single,
    frameround=tttt,
    rulecolor=\color{code-border},
    xleftmargin=5pt,
    xrightmargin=5pt,
    upquote=true,
    columns=fixed,
    extendedchars=true,
    inputencoding=utf8,
    literate=
        {á}{{\'a}}1 {é}{{\'e}}1 {í}{{\'i}}1 {ó}{{\'o}}1 {ú}{{\'u}}1
        {Á}{{\'A}}1 {É}{{\'E}}1 {Í}{{\'I}}1 {Ó}{{\'O}}1 {Ú}{{\'U}}1
        {ñ}{{\~n}}1 {Ñ}{{\~N}}1
        {ü}{{\"u}}1 {Ü}{{\"U}}1
        {¿}{{?`}}1 {¡}{{!`}}1
}

\lstset{style=up-code}
%%%%%%%%%%%%%%%%%%%%%%%%%%%%%%%%%%%%%%%%%%%%%%%%%%%%%%%%%%%%%%%%%%%%%%%%%%%%%%%%


% Para poder hacer flechas
\usetikzlibrary{shapes, arrows}

% Sección de definiciones
\titleformat{\section}{\Large\bfseries}{\thesection}{1em}{}
\titleformat{\subsection}{\large\bfseries}{\thesubsection}{1em}{}

% Caja de colores
\definecolor{mint}{RGB}{202,251,202}
\definecolor{yellow}{RGB}{255,255,202}
\definecolor{red}{RGB}{255,202,202}

% Variables globales para el documento
\newcommand{\universityname}{Universidad de Palermo}
\newcommand{\facultyname}{Facultad de Ingeniería}
\newcommand{\coursename}{\textbf{Algoritmos 1}}
\newcommand{\currentsemester}{Primer Semestre}
\newcommand{\currentyear}{2026}
\newcommand{\copyrightnotice}{
  \scriptsize \textcopyright{} \currentyear \space \universityname. Prohibida la reproducción total o parcial de imágenes y textos
}

% Esto es para poder agregar comentarios al código
\newcommand{\comentario}[1]{\textcolor{gray}{// #1}}

% Definir los encabezados y pies de página
\pagestyle{fancy}
\fancyhf{} % Borra encabezados y pies de página

% Configuración del header
\fancyhead[R]{\includegraphics[height=50px]{../latex/images/logo_up.jpg}}
\renewcommand{\headrulewidth}{0.4pt} % Agregar línea debajo del encabezado
\fancyhead[L]{\scriptsize \facultyname}
\fancyhead[C]{\small \coursename}
\setlength{\headheight}{50pt} % Ajustar la altura del encabezado

% Configuración del footer
\fancyfoot[C]{\copyrightnotice}
\fancyfoot[R]{\thepage}
\renewcommand{\footrulewidth}{0.4pt} % Agregar línea arriba del pie de página
\setlength{\footskip}{40pt} % Agregar separación entre el texto y el pie de página

% Configuración de la primera página
\fancypagestyle{firstpage}{
  \fancyhf{} % Borra encabezados y pies de página
  \fancyfoot[C]{\thepage} % Número de página centrado
  \renewcommand{\headrulewidth}{0pt} % Sin línea en el encabezado
  \renewcommand{\footrulewidth}{0.4pt} % Sin línea en el pie de página
}

% Configuración de la primera página
\AtBeginDocument{\thispagestyle{firstpage}}
