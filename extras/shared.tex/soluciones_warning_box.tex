\begin{tcolorbox}[colback=red,colframe=red!75!black,arc=0pt,outer arc=0pt]
  \textbf{¡STOP!} \\

Estas son las ¡SOLUCIONES! a los problemas \textbf{¿estás seguro que vas a revisar esto?} \\
La idea de estas soluciones es que estén aquí para que puedas revisarlas cuando ya agotaste todas las posibilidades de 
poder resolver el problema por tu cuenta.

Nuestra recomendaciones son las siguientes:

\begin{itemize}
  \item \textbf{NO leas la solución de un problema sin intentarlo antes}: Intentar (y fallar) es parte del proceso de
    aprendizaje. Si leés (o incluso si sólo la mirás) vas a intentar llegar a esa solución y no resolver el problema.

\item \textbf{¡Lo intenté y no me sale!}: El momento de real aprendizaje ocurre cuando intentás suficiente y, de
  repente, te sale. Es ese momento donde decís en voz alta \textbf{¡AHA!} y entendiste cómo resolver el problema.
    Siempre intentá un poquito más antes de leer las soluciones.

\item \textbf{Ya lo pensé de varias formas diferentes... ¡pero no me sale!}: Si creés
  que lo intentaste suficiente y aún así no te salió, es el momento de leer la solución. \textbf{Sin embargo}, no leas
    la solución completa, intentá primero leer las pistas (si las hay) y ver si con eso podés llegar a la solución.

\item \textbf{¡Leí todas las pistas y aún así no me doy cuenta!}: No te preocupes, a veces pasa. Este es el momento de
  leer la solución completa. En ese momento, te recomendamos que sigas el siguiente proceso: \textbf{1. lee la solución
    completa, 2. dejá el problema un tiempo, 3. intenta resolver el problema sin mirar la solución}. La idea acá es que
    no copies inmediatamente la solución y que intentes no acordártela de memoria, sino que uses ese conocimiento para
    hacer tu propio razonamiento y logres implementar la solución por tu cuenta.

  \textbf{¡Éxitos en la práctica!}
\end{itemize}
\end{tcolorbox}
