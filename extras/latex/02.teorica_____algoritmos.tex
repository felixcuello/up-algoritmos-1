% \documentclass[10pt, twocolumn, a4paper]{article}
\documentclass[12pt,a4paper]{article}

\usepackage[utf8]{inputenc}                                             % To handle UTF-8 input
\usepackage[T1]{fontenc}                                                % Better font encoding that supports accents
\usepackage[backend=biber, style=ieee]{biblatex}                        % To include the bibliography
\usepackage[left=2cm, right=2cm, top=2.5cm, bottom=4cm]{geometry}     % To set the margins
\usepackage[noend]{algpseudocode}
\usepackage[table]{xcolor}                                              % For coloring cells

\usepackage{algorithm}                                                  % To include algorithms
\usepackage{amsfonts}                                                   % To include math fonts:ToggleTerm direction=float
\usepackage{amsmath}                                                    % To include Mathematic symbols
\usepackage{authblk}                                                    % To format author affiliations
\usepackage{caption}                                                    % For caption spacing
\usepackage{enumitem}                                                   % To customize lists (items like i, ii, iii, iv)
\usepackage{float}                                                      % To place figures where you want them
\usepackage{fancyhdr}                                                   % To customize headers and footers
\usepackage{graphicx}                                                   % To include images
\usepackage{hyperref}                                                   % To include hyperlinks
\usepackage{lipsum}                                                     % TODO: remove this
\usepackage{pgfplots}                                                   % To plot functions
\usepackage{listings}                                                   % To include code
\usepackage{tabularx}                                                   % For equal-width columns
\usepackage{tcolorbox}                                                  % To make colored boxes
\usepackage{tikz}                                                       % To draw graphs
\usepackage{titlesec}                                                   % To format section titles
\usepackage{xcolor}                                                     % To define colors

\usetikzlibrary{graphs,graphs.standard, arrows.meta}
\usetikzlibrary{positioning}

\addbibresource{./references.bib}

%%%%%%%%%%%%%%%%%%%%%%%%%%%%%%%%%%%%%%%%%%%%%%%%%%%%%%%%%%%%%%%%%%%%%%%%%%%%%%%%
% Estilo de código
%%%%%%%%%%%%%%%%%%%%%%%%%%%%%%%%%%%%%%%%%%%%%%%%%%%%%%%%%%%%%%%%%%%%%%%%%%%%%%%%
\definecolor{up-blue}{RGB}{0,102,254}
\definecolor{up-violet}{RGB}{100,30,180}
\definecolor{up-gray}{RGB}{102,102,102}
\definecolor{up-red}{RGB}{152,102,52}
\definecolor{up-comment}{RGB}{153,153,153}
\definecolor{code-bg}{RGB}{248,248,248}
\definecolor{code-border}{RGB}{104,104,104}

\lstdefinestyle{up-code}{
    backgroundcolor=\color{code-bg},
    commentstyle=\color{up-comment}\itshape\footnotesize,
    keywordstyle=\color{up-blue}\bfseries,
    numberstyle=\tiny\color{up-red},
    stringstyle=\color{up-violet},
    basicstyle=\ttfamily\footnotesize,
    breakatwhitespace=false,
    breaklines=true,
    captionpos=b,
    keepspaces=true,
    numbers=left,
    numbersep=5pt,
    showspaces=false,
    showstringspaces=false,
    showtabs=false,
    tabsize=2,
    frame=single,
    frameround=tttt,
    rulecolor=\color{code-border},
    xleftmargin=5pt,
    xrightmargin=5pt,
    upquote=true,
    columns=fixed,
    extendedchars=true,
    inputencoding=utf8,
    literate=
        {á}{{\'a}}1 {é}{{\'e}}1 {í}{{\'i}}1 {ó}{{\'o}}1 {ú}{{\'u}}1
        {Á}{{\'A}}1 {É}{{\'E}}1 {Í}{{\'I}}1 {Ó}{{\'O}}1 {Ú}{{\'U}}1
        {ñ}{{\~n}}1 {Ñ}{{\~N}}1
        {ü}{{\"u}}1 {Ü}{{\"U}}1
        {¿}{{?`}}1 {¡}{{!`}}1
}

\lstset{style=up-code}
%%%%%%%%%%%%%%%%%%%%%%%%%%%%%%%%%%%%%%%%%%%%%%%%%%%%%%%%%%%%%%%%%%%%%%%%%%%%%%%%


% Para poder hacer flechas
\usetikzlibrary{shapes, arrows}

% Sección de definiciones
\titleformat{\section}{\Large\bfseries}{\thesection}{1em}{}
\titleformat{\subsection}{\large\bfseries}{\thesubsection}{1em}{}

% Caja de colores
\definecolor{mint}{RGB}{202,251,202}
\definecolor{yellow}{RGB}{255,255,202}
\definecolor{red}{RGB}{255,202,202}

% Variables globales para el documento
\newcommand{\universityname}{Universidad de Palermo}
\newcommand{\facultyname}{Facultad de Ingeniería}
\newcommand{\coursename}{\textbf{Algoritmos 1}}
\newcommand{\currentsemester}{Primer Semestre}
\newcommand{\currentyear}{2026}
\newcommand{\copyrightnotice}{
  \scriptsize \textcopyright{} \currentyear \space \universityname. Prohibida la reproducción total o parcial de imágenes y textos
}

% Esto es para poder agregar comentarios al código
\newcommand{\comentario}[1]{\textcolor{gray}{// #1}}

% Definir los encabezados y pies de página
\pagestyle{fancy}
\fancyhf{} % Borra encabezados y pies de página

% Configuración del header
\fancyhead[R]{\includegraphics[height=50px]{../latex/images/logo_up.jpg}}
\renewcommand{\headrulewidth}{0.4pt} % Agregar línea debajo del encabezado
\fancyhead[L]{\scriptsize \facultyname}
\fancyhead[C]{\small \coursename}
\setlength{\headheight}{50pt} % Ajustar la altura del encabezado

% Configuración del footer
\fancyfoot[C]{\copyrightnotice}
\fancyfoot[R]{\thepage}
\renewcommand{\footrulewidth}{0.4pt} % Agregar línea arriba del pie de página
\setlength{\footskip}{40pt} % Agregar separación entre el texto y el pie de página

% Configuración de la primera página
\fancypagestyle{firstpage}{
  \fancyhf{} % Borra encabezados y pies de página
  \fancyfoot[C]{\thepage} % Número de página centrado
  \renewcommand{\headrulewidth}{0pt} % Sin línea en el encabezado
  \renewcommand{\footrulewidth}{0.4pt} % Sin línea en el pie de página
}

% Configuración de la primera página
\AtBeginDocument{\thispagestyle{firstpage}}


\begin{document}

\begin{center}
  \LARGE\textbf{\coursename} \\
  \Large{Teórica 02 - Algoritmos y Estructura de Datos} \\
  \normalsize{\currentsemester, \currentyear} \\
  \vspace{1em}
  \hrule
\end{center}

\vspace{1em}

\setcounter{section}{2}

\newpage

\tableofcontents

\newpage

\subsection{Introducción}
\label{sec:introduccion}

La teórica de algoritmos se enfoca a la resolución de problemas algorítmicos. La razón por la cual es importante
estudiar algoritmos, es que necesitamos aprender a resolver problemas con CPUs de manera eficiente para poder luego
empezar a pensar los problemas de una manera paralela y distribuida con GPUs.

Las técnicas para resolver algoritmos pueden variar dependiendo del problema, y sólo vamos a ver algunas de las más
sencillas para que puedan empezar a pensar en cómo resolverlos. Otras técnicas más avanzadas las verán en materias como
\textbf{Algoritmos 1} y \textbf{Algoritmos 2}, que son materias enfocadas a exclusivamente al estudio de algoritmos y
cómo implementarlos.

Es por eso que esta teórica no pretende ser exhaustiva, sino que es una introducción a los conceptos básicos que
necesitaremos para la materia.

\textbf{Definición de algoritmo:} Un algoritmo es un conjunto finito y ordenado de pasos o instrucciones que se siguen
para resolver un problema o realizar una tarea específica. Cada paso debe ser claro, preciso y ejecutable, y el
algoritmo debe llevar a una solución definida en un número finito de pasos.

Gran parte de esta teórica está basada en el libro "Guide to Competitive Programming" \cite{gcc2017} y "Cracking the
Coding Interview" \cite{cracking2015} que son libros de textos para estudiantes de computación que se enfocan en la
solución de problemas algorítmicos para competencias de programación y la preparación para entrevistas técnicas en
empresas de tecnología. Ambos libros son excelentes recursos en el caso de que quieran profundizar en el tema de
algoritmos y aprender a pasar entrevistas técnicas. \footnote{Recuerden siempre que tener el conocimiento para resolver
problemas algoritmos es el primer paso para conseguir un trabajo en empresas de tecnología, particularmente en FAANG
(Facebook, Amazon, Apple, Netflix y Google).}

\subsection{Números}
\label{sec:numeros}

\subsubsection{Enteros}

C posee varios tipos de datos para representar números. El más utilizado es el tipo \texttt{int}, que tiene 32 bits y
representa números enteros en el rango de \(-2^{31}\) a \(2^{31}-1\). Cuando \texttt{int} no es suficiente, se puede
utilizar el tipo \texttt{long long}, que tiene 64 bits y representa números enteros en el rango de \(-2^{63}\) a
\(2^{63}-1\). De todas formas siempre se puede agregar el modificador \texttt{unsigned} para representar números enteros
sin signo, lo que permite representar números enteros en el rango de \(0\) a \(2^{32}-1\) para \texttt{unsigned int} y
\(0\) a \(2^{64}-1\) para \texttt{unsigned long long}.

\newpage % ------------------------------------------------------------------------------------------------------------

\subsubsection{Aritmética modular}
\label{sec:aritmetica_modular}

A veces necesitamos trabajar con números que se van de los rangos de los tipos de datos. En estos casos, podemos
utilizar la aritmética modular, que nos permite trabajar con números en un rango específico. El módulo (operación
representada por \texttt{\%}) es una operación que nos da el resto de la división. Por ejemplo, \texttt{5 \% 3 == 2},
porque \(5\) dividido por \(3\) da \(1\) con un resto de \(2\). En general si \texttt{f(n)} es una función que devuelve
el un valor entero, \texttt{f(n) \% m} nos va a dar un número en el rango de \(0\) a \(m-1\), sin importar el valor de
retorno de \texttt{f(n)}.

\subsubsection{Flotantes}
El tipo \texttt{float} en C representa números de punto flotante de precisión simple, mientras que el tipo
\texttt{double} representa números de punto flotante de precisión doble. Estos tipos son útiles para representar números
con decimales, pero es importante tener en cuenta que pueden introducir errores de redondeo debido a la forma en que se
almacenan los números en la memoria \footnote{\href{https://en.wikipedia.org/wiki/IEEE_754}{IEEE 754}}.

Con lo cual, supongamos que tenemos el siguiente código:

\begin{lstlisting}[language=C]
double x = 0.3*3+0.1;
printf("%.20f\n", x); // 0.99999999999999988898
\end{lstlisting}

Esto se debe a que \(0.3\) y \(0.1\) no pueden ser representados exactamente en binario, lo que provoca un error de
redondeo al realizar la operación. Esto no es exclusivo de C, sino que es un problema de la representación de números de
punto flotante en general.

Con lo cual muchas veces es mejor trabajar con números enteros, o si es necesario trabajar con números de punto flotante
es preferible utilizar umbrales de tolerancia, por ejemplo:

\begin{lstlisting}[language=C]
#include <math.h> // Recordar incluir esta librería para poder usar fabs(double)

if(fabs(a-b) < 1e-9) {
  // a y b son considerados iguales
}
\end{lstlisting}

\subsection{Macros}
\label{sec:macros}

Las macros son una herramienta poderosa en C que nos permite definir constantes y algunas funciones cuando necesitamos
repetir código. Por ejemplo, podemos definir una macro para el valor de \(\pi\) o una función para calcular el máximo y
mínimo entre dos números. Las macros se definen utilizando la directiva \texttt{\#define} y se pueden utilizar en
cualquier parte del código, pero normalmente se definen al principio del archivo (para más prolijidad):

\begin{lstlisting}[language=C]
#define PI 3.14159265358979323846
#define MAX(a, b) ((a) > (b) ? (a) : (b))
#define MIN(a, b) ((a) < (b) ? (a) : (b))
\end{lstlisting}

\subsection{Manipulación de bits}

En programación, los números enteros se representan con n-bits, donde n es el número de bits que se utilizan para
almacenarlos. Por ejemplo, un número entero de 32 bits se representa con 32 bits en la memoria. Cada bit puede ser 0 o
1, lo que significa que un número entero de 32 bits puede representar \(2^{32}\) valores diferentes. Por ejemplo el
número \texttt{43} se representa en binario como \texttt{00000000000000000000000000101011}.

Genéricamente la fórmula para convertir un número decimal en binario es:

\[
  d = 2^{n-1} \cdot b_{n-1} + 2^{n-2} \cdot b_{n-2} + \ldots + 2^1 \cdot b_1 + 2^0 \cdot b_0
\]

Entonces por ejemplo el número \(43\) se representa en binario como:

\[
  43 = 1 \cdot 2^5 + 0 \cdot 2^4 + 1 \cdot 2^3 + 0 \cdot 2^2 + 1 \cdot 2^1 + 1 \cdot 2^0
\]

Los números negativos se representan utilizando el \textit{complemento a dos}, que se calcula invirtiendo todos los bits
del número y sumando \(1\) al resultado. Por ejemplo, el número \(-43\) se representa en binario como:

\[
  -43 = 11111111111111111111111111010101
\]

En C, el operador de complemento a dos se representa con el operador \texttt{~}.

\subsection{Operaciones con bits}

En C se pueden hacer operaciones sobre bits utilizando los operadores \texttt{\&}, \texttt{|}, \texttt{\string^},
\texttt{\textasciitilde}, \texttt{{<}{<}} y \texttt{{>}{>}}.

\textbf{Operación AND} produce un 1 en el bit de salida si ambos bits de entrada son 1. Por ejemplo:

\[
\begin{array}{r@{\;}l}
   10110 & (22) \\
   \texttt{\&}\;11010 & (26) \\
   \hline
   10010 & (18) \\
\end{array}
\]

\textbf{Operación OR} produce un 1 en el bit de salida si al menos uno de los bits de entrada es 1. Por ejemplo:

\[
\begin{array}{r@{\;}l}
   10110 & (22) \\
   \texttt{|}\;11010 & (26) \\
   \hline
   11110 & (30) \\
 \end{array}
\]

\textbf{Operación XOR} produce un 1 en el bit de salida si los bits de entrada son diferentes. Por ejemplo:

\[
\begin{array}{r@{\;}l}
   10110 & (22) \\
   \texttt{\string^}\;11010 & (26) \\
   \hline
   01100 & (12) \\
 \end{array}
\]

\textbf{Operación NOT} invierte los bits de entrada. Por ejemplo:

\[
\begin{array}{r@{\;}l}
  \verb|~| 43 = 11111111111111111111111111010100 = 4294967252 \\
\end{array}
\]

\textbf{Operación SHIFT} desplaza los bits de entrada a la izquierda o a la derecha. Esto es análogo a multiplicar o
hacer la división entera por 2. Por ejemplo:

\[
\begin{array}{r@{\;}l}
  43 << 1 = 00000000000000000000000001010110 = 86 \\
  43 >> 1 = 00000000000000000000000000001011 = 21 \\
\end{array}
\]

\textbf{Máscaras} se utilizan para seleccionar bits específicos de un número. Por ejemplo, si se quiere seleccionar
los bits 4 a 7 de un número, se puede hacer lo siguiente:

\[
\begin{array}{r@{\;}l}
  43 & = 00000000000000000000000000101011 \\
  \text{mask} & = 00000000000000000000000000001111 \\
  \hline
  43 \& \text{mask} & = 00000000000000000000000000001011 = 11 \\
\end{array}
\]

\subsection{Representación de Conjuntos}

Cada subconjunto del conjunto \(\{0, 1, 2, \ldots, n-1\}\) se puede representar como un número entero de \(n\) bits,
donde cada bit representa si el elemento está presente en el subconjunto o no. Por ejemplo:

\begin{lstlisting}[language=C]
int x = 0;
x |= (1 << 1); // Agregar el elemento 1 al conjunto
x |= (1 << 3); // Agregar el elemento 3 al conjunto
x |= (1 << 4); // Agregar el elemento 4 al conjunto
x |= (1 << 8); // Agregar el elemento 8 al conjunto

for(int i = 0; i < 32; i++) {
  if(x & (1 << i)) {
    printf("%d ", i); // Imprimir los elementos del conjunto
  }
}
\end{lstlisting}

\newpage % ------------------------------------------------------------------------------------------------------------

\subsection{Eficiencia algorítmica}

Esto lo vamos a ver con más detalle en la próxima teórica, pero es importante mencionar que la eficiencia algorítmica
estima cuánto tiempo va a utilizar un algoritmo para resolver un problema en función del tamaño de la entrada.

La complejidad algorítimica se nota con la notación \(\mathcal{O}\) (grande O), que estima la cota superior del tiempo
de ejecución de un algoritmo en función del tamaño de la entrada. La cota superior es la estimación del peor caso
posible en base al tamaño de la entrada.

\textbf{Ejemplos}:

\begin{itemize}
  \item \(\mathcal{O}(1)\): una asignación o una cuenta como $\frac{1}{2} + \frac{1}{3}$, el tiempo de
    ejecución no depende del tamaño de la entrada.
  \item \(\mathcal{O}(\log n)\): una búsqueda binaria, el tiempo de ejecución crece logarítmicamente
    con el tamaño de la entrada.
  \item \(\mathcal{O}(n)\): recorrer un arreglo de \(n\) elementos.
  \item \(\mathcal{O}(n \log n)\): un algoritmo de ordenamiento eficiente como \texttt{quicksort} o \texttt{mergesort}.
  \item \(\mathcal{O}(n^2)\): buscar todos los pares de elementos que cumplen con una cierta condición
    en un arreglo de \(n\) elemento.
  \item \(\mathcal{O}(2^n)\): Generar todos los subconjuntos de un conjunto de \(n\) elementos.
  \item \(\mathcal{O}(n!)\): Resolver el problema de las n-reinas, que consiste en colocar \(n\) reinas en un
    tablero de ajedrez de \(n \times n\) de tal manera que ninguna reina ataque a otra.
\end{itemize}

Estimar la eficiencia algorítmica de un algoritmo es algo que puede revisarse \textit{antes} de implementarlo, y es una
herramienta muy útil para saber si la solución que estamos pensando es viable o no.

\textbf{Formalmente}: que un algoritmo funcione en $O(f(n))$ significa que existe una constante $c$ y un valor $n_0$ tal
que para todo $n \geq n_0$ se cumple que el algoritmo va a realizar como máximo $c \cdot f(n)$ operaciones para todas
las operaciones donde $n \geq n_0$. De todas formas se intenta utilizar el mejor caso posible de límite superior, es
decir que si un algoritmo funciona en $O(n)$, su límite también podría ser $O(n^2)$, pero se prefiere utilizar el límite
más bajo posible.


\newpage % ------------------------------------------------------------------------------------------------------------

\subsection{Soluciones a los algoritmos}

Los algoritmos presentados en las soluciones no están escritos en un lenguaje de programación específico, sino que están
escritos en un pseudocódigo que es fácil de entender e implementar en cualquier lenguaje de programación. La idea es que
por un lado aprendan a leer algoritmos y por otro lado aprendan a escribir código sin copiar y pegar.

Por ejemplo imaginemos que queremos escribir un algoritmo que implemente la
\href{https://es.wikipedia.org/wiki/Conjetura_de_Collatz}{Conjetura de Collatz}. En pocas palabras, la conjetura de
Collatz determina que una función $f(n)$ que se define de la siguiente manera:

\[
f(n) =
\begin{cases} 
\frac{n}{2} & \text{si } n \text{ es par} \\
3n + 1 & \text{si } n \text{ es impar}
\end{cases}
\]

Por ejemplo:

\begin{tikzpicture}[
  grow=left,
  level distance=1.5cm,
  level 1/.style={sibling distance=2cm},
  level 2/.style={sibling distance=1cm},
  level 3/.style={sibling distance=1cm},
  level 4/.style={sibling distance=1cm},
  every node/.style = {shape=rectangle, rounded corners,
    draw, align=center,
    top color=white, bottom color=blue!20}]
  \node {1}
    child { node {2}
      child { node {4}
        child { node {8}
          child { node {16}
            child { node {5} }
            child { node {32}
              child { node {64}
                child { node {21} }
              }
            }
          }
        }
      }
    };
\end{tikzpicture}

Se puede pensar en el siguiente algoritmo para resolver la conjetura de Collatz:

\begin{algorithm}
\caption{Conjetura de Collatz}
  \begin{algorithmic}[1]
    \Statex \textbf{Define:} $f(n)$
    \Statex \textbf{Input:} $n \in \mathbb{N}$
    \Statex \textbf{Initialization:} $\emptyset$
    \Statex \textbf{Output:} Los números de la conjetura de Collatz hasta llegar a 1

    \Function{collatz}{$n$}
      \While{$n \neq 1$}
        \State \textbf{print} $n$
        \If{$n$ is odd}
          \State $n \gets n / 2$
        \Else
          \State $n \gets 3n + 1$
        \EndIf
      \EndWhile

      \State \textbf{print} $n$
    \EndFunction
  \end{algorithmic}
\end{algorithm}

\newpage % ------------------------------------------------------------------------------------------------------------

\subsection{Teoría de Números}

La teoría de númers es la rama de las matemáticas que estudia los números enteros. En esta sección vamos a charlar un
poco sobre algunos conceptos de algoritmos y teoría de números.

\subsection{Números primos}

Se dice que un número entero $a$ es un \textit{factor} o un \textit{divisor} de un número entero $b$, si $a$ divide a
$b$. Esto significa que existe un número entero $c$ tal que $b = a \cdot c$. Se nota como $a \mid b$. Por ejemplo los
factores de $24$ son $1$, $2$, $3$, $4$, $6$, $8$, $12$ y $24$.

Un número entero $n$ > 1 es \textit{primo} si sus únicos factores son $1$ y $n$. Por ejemplo los números $2$, $3$, $5$,
$7$, $11$ son primos, mientras que $93$ no es primo ya que $93 = 3 \cdot 31$. Además para cada entero $n \ge 1$ existe
una \textit{única factorización} de $n$ en números primos:

\[
  n = p_1^{a_1} \cdot p_2^{a_2} \cdots p_k^{a_k}
\]

Por ejemplo:

\[
  84 = 2^2 \cdot 3^1 \cdot 7^1
\]

Los algoritmos de búsqueda de números primos son bastante lentos e incluso las formas más complejas de encontrar si un
número es primo o no, como el test de primalidad de AKS
\footnote{\href{https://es.wikipedia.org/wiki/Test_de_primalidad_AKS}{Test de primalidad AKS}}, son muy lentos para
primos muy grandes (números de 512 bits o más).

Supongamos que queremos saber si un número \(n\) es primo o no, veamos los algoritmos más sencillos para resolver este
problema.

\subsubsection{Búsqueda lineal}
Este es el algoritmo más simple y consiste en recorrer todos los números desde \(2\) hasta \(n\) y verificar si son
primos o no. Si bien el algoritmo no es muy eficiente, es fácil de entender y de implementar.

\begin{lstlisting}[language=C]
bool es_primo(int n) {
    if (n <= 1) return false; // no es primo porque 
    for (int i = 2; i <= n; i++)
        if (n % i == 0) return false; // No es primo
    return true; // Es primo
}
\end{lstlisting}


\subsubsection{Búsqueda lineal hasta $\sqrt{n}$}
Si pensamos un poco el algoritmo anterior nos vamos a dar cuenta de que no es necesario recorrer todos los números hasta
$n$. Sin embargo nos podemos preguntar \textbf{¿Hasta qué número tenemos que recorrer para saber si un número es primo o
no?}, la respuesta no es inmeditamente trivial.

Claramente $n-1$ nunca puede ser un divisor de $n$ (excepto en el caso $n=2$), pero entonces ¿es divisor $n-2, n-3,
\cdots, n-k$?

Si $n$ no es primo (por lo que dijimos antes) debe tener por lo menos dos factores $a$ y $b$ tales que $n = a \cdot b$.
Se puede demostrar fácilmente que al menos uno de esos factores debe ser menor o igual a $\sqrt{n}$. \\

\textbf{Demostración:}
Sea \( n \in \mathbb{N} \). Si \( n \) no es primo, entonces tiene al menos dos factores positivos distintos, digamos \(
a \) y \( b \), tales que \( a \cdot b = n \). Supongamos que ambos \( a \) y \( b \) son mayores que \( \sqrt{n} \).
Entonces se tendría que \( a > \sqrt{n} \) y \( b > \sqrt{n} \), lo que implica que \( ab > \sqrt{n} \cdot \sqrt{n} = n
\), una contradicción. Por lo tanto, al menos uno de los factores debe ser menor o igual a \( \sqrt{n} \). Así, si \( n
\) tiene un divisor propio (por ejemplo, un primo que lo divide), debe encontrarse entre 2 y \( \sqrt{n} \). \\

\textbf{Código:}

\begin{lstlisting}[language=C]
bool es_primo(int n) {
    if (n <= 1) return false; // no es primo porque 
    for (int i = 2; i * i <= n; i++)
        if (n % i == 0) return false; // No es primo
    return true; // Es primo
}
\end{lstlisting}

Notar que hacer que el algoritmo recorra hasta $i * i \leq n$ es lo mismo que recorrer hasta $i \leq \sqrt{n}$, con la
ventaja de que no necesitamos utilizar números de punto flotante para calcular la raíz cuadrada, lo que evita errores de
redondeo.


\subsubsection{Criba de Eratóstenes}
Una forma muy eficiente de encontrar números primos es utilizar la Criba de Eratóstenes. Pero tiene la desventaja de que
necesitamos tener muchísima memoria disponible, ya que utlizaremos un vector de booleanos de tamaño \(n\) para saber si
el número es primo o no. El algoritmo consiste en recorrer todos los números desde \(2\) hasta \(n\) y e ir marcando los
números compuestos como NO primos. Cuando terminamos de recorrer todos los números vamos a tener un listado de números
primos entre \(2\) y \(n\).


\begin{lstlisting}[language=C]
#include <stdio.h>
#include <stdbool.h>

int main() {
  int n = 100; // Cambiar este valor para encontrar primos hasta n
  bool es_primo[n + 1];
  for (int i = 0; i <= n; i++) es_primo[i] = true; // Inicializar el vector

  es_primo[0] = es_primo[1] = false; // 0 y 1 no son primos

  for (int i = 2; i * i <= n; i++) {
      if (es_primo[i]) { // Si es primo
          for (int j = i * i; j <= n; j += i) { // Marcar todos los multiplos de i como NO primos
              es_primo[j] = false;
          }
      }
  }

  // Imprimir los numeros primos
  printf("Numeros primos hasta %d:\n", n);
  for (int i = 2; i <= n; i++) {
      if (es_primo[i]) printf("%d ", i);
  }
}
\end{lstlisting}

Hay otros métodos más avanzados para encontrar números primos como \textit{Pollard Rho} o el \textit{Test de
Miller-Rabin}, pero exceden lárgamente el alcance de esta teórica.

\subsection{Vectores}

Los arrays en C son porciones de memoria contiguas que almacenan elementos del mismo tipo. Se pueden declarar de la
siguiente manera:

\begin{lstlisting}[language=C]
int arr[10]; // Un array de 10 enteros
int arr15] = {1, 2, 3, 4, 5}; // Un array de 5 enteros inicializado
int arr2] = {1, 2, 3, 4, 5}; // Un array de 5 enteros inicializado
int *arr3= malloc(10 * sizeof(int)); // Un array de 10 enteros dinamico
\end{lstlisting}

Los arrays se pueden recorrer utilizando un bucle \texttt{for} o \texttt{while}, y se accede a sus elementos utilizando
el operador de índice \texttt{[]}. Por ejemplo, para recorrer un array de enteros y sumar sus elementos:

\begin{lstlisting}[language=C]
#include <stdio.h>
#include <stdlib.h>

int main() {
    int n = 5;
    int *arr = malloc(n * sizeof(int));
    for (int i = 0; i < n; i++) {
        arr[i] = i + 1; // Inicializar el array
    }

    int sum = 0;
    for (int i = 0; i < n; i++) {
        sum += arr[i]; // Sumar los elementos del array
    }

    printf("La suma de los elementos del array es: %d\n", sum);
    free(arr); // Liberar la memoria del array
    return 0;
}
\end{lstlisting}

\newpage % ------------------------------------------------------------------------------------------------------------

\subsection{Matrices}

Las matrices son un concepto matemático que se utiliza para representar datos en forma de tablas. En C, las matrices se
pueden declarar como arrays bidimensionales; o como veremos más adelante, como arrays unidimensionales donde cada
elemento de la matriz se calcula como un índice en el array unidimensional.

Por ejemplo una matriz de \(3 \times 3\) se puede declarar de la siguiente manera:

\begin{lstlisting}[language=C]
int matriz[3][3]; // matriz de 3x3 enteros
int matriz2[3][3] = {
    {1, 2, 3},
    {4, 5, 6},
    {7, 8, 9}
}; // matriz de 3x3 enteros inicializada

int *matriz3 = (int *)malloc(3 * 3 * sizeof(int)); // matriz de 3x3 enteros dinamica
\end{lstlisting}

Por otro lado un vector unidimensional de \(n\) elementos es como si fuera una matriz de \(1 \times n\) o \(n \times
1\), por lo que se puede declarar de la siguiente manera:

\begin{lstlisting}[language=C]
int vector[10]; // Vector de 10 enteros
int vector2[10] = {1, 2, 3, 4, 5, 6, 7, 8, 9, 10}; // Vector de 10 enteros inicializado
int *vector3 = (int *)malloc(10 * sizeof(int)); // Vector de 10 enteros dinamico
\end{lstlisting}

\subsubsection{Acceso a los elementos de una matriz}

Cuando se trabajan con índices en un vector unidimensional se suele utilizar el índice \texttt{i} para hablar del
i-ésimo elemento del vector. Lamentablemente, por analogía, en una matriz bidimensional se suelen utilizar los índices
\texttt{i} y \texttt{j} para hablar de las filas y las columnas de la matriz, lo que puede llevar a confusiones. En el
libro Cracking The Coding Interview \cite{cracking2015} se recomiendan la utilización de los índices \texttt{row} y
\texttt{col} para hablar de las filas y las columnas de una matriz, lo que ayuda a evitar confusiones, y es lo que
utilizaremos en la materia (a veces también \texttt{fila} y \texttt{col}).

Sin embargo, como veremos en la práctica de CUDA, a veces no podemos saber el tamaño de la matriz en tiempo de
compilación y tenemos que utilizar un array unidimensional para representar la matriz. En este caso, el elemento
\texttt{(row, col)} de la matriz se puede calcular como:

\[
  \text{matriz}[row][col] = \text{array}[row \cdot n + col]
\]

Donde n es el número de columnas de la matriz. Por ejemplo si tenemos una matriz de $5$ filas y $10$ columnas y queremos
accedeer al número en la posición \texttt{matrix[2][3]} (es decir, fila $2$ y columna $3$), podemos calcular el índice
de la siguiente manera:

\[
  \text{matrix}[2][3] = \text{array}[2 \cdot 10 + 3] = \text{array}[20 + 3] = \text{array}[23]
\]

\subsubsection{Suma de matrices}

La suma de dos matrices $A + B$ está definida sólo para matrices del mismo tamaño. El resultado es una matriz donde cada
elemento es la suma de los elementos correspondientes de las dos matrices, por ejemplo si tenemos las matrices:

\[
  \begin{bmatrix}
  a_{11} & a_{12} & a_{13} \\
  a_{21} & a_{22} & a_{23}
  \end{bmatrix}
  +
  \begin{bmatrix}
  b_{11} & b_{12} & b_{13} \\
  b_{21} & b_{22} & b_{23}
  \end{bmatrix}
  =
  \begin{bmatrix}
  a_{11}+b_{11} & a_{12}+b_{12} & a_{13}+b_{13} \\
  a_{21}+b_{21} & a_{22}+b_{22} & a_{23}+b_{23}
  \end{bmatrix}
\]

\subsubsection{Multiplicación de matrices}

La multiplicación de matrices $A \cdot B$ está definida sólo para matrices donde el número de columnas de la primera
matriz es igual al número de filas de la segunda matriz. El resultado se calcula con la siguiente fórmula:

\[
  (A \cdot B)_{ij} = \sum_{k=1}^{n} A_{i, k} \cdot B_{k, j}
\]

(Sí, en este caso utilizamos $i$ y $j$ para las filas y columnas de la matriz, ya que es una convención común en
álgebra, si fuera código recomendaríamos utilizar \texttt{row\_a, col\_a} y \texttt{row\_b, col\_b}.

Si tuviéramos dos matrices $A$ y $B$ de tamaño $n \times n$, el código tendría la siguiente forma:

\begin{lstlisting}[language=C]
for (int i = 1; i <= n; i++) {
  for (int j = 1; j <= n; j++) {
    for (int k = 1; k <= n; k++) {
      C[i][j] += A[i][k] * B[k][j];
    }
  }
}
\end{lstlisting}

\newpage % ------------------------------------------------------------------------------------------------------------

\subsubsection{Flip de matrices}

Definimos el flip de una matriz como la operación que intercambia las filas y las columnas de la matriz. Es decir, que
dada una matriz $A$ el flip horizontal se define como $swap(A_{i, j}) = A_{n - i, j}) \forall i, j$ y el flip vertical se
define como $swap(A_{i, j}) = A_{i, m - j}) \forall i, j$ con $n$ el número de filas y $m$ el número de columnas.

\[
  \begin{bmatrix}
  a_{11} & a_{12} & a_{13} \\
  a_{21} & a_{22} & a_{23}
  \end{bmatrix}
  \xrightarrow{\text{flip vertical}}
  \begin{bmatrix}
  a_{21} & a_{22} & a_{23} \\
  a_{11} & a_{12} & a_{13}
  \end{bmatrix}
\]

\[
  \begin{bmatrix}
  a_{11} & a_{12} & a_{13} \\
  a_{21} & a_{22} & a_{23}
  \end{bmatrix}
  \xrightarrow{\text{flip horizontal}}
  \begin{bmatrix}
  a_{13} & a_{12} & a_{11} \\
  a_{23} & a_{22} & a_{21}
  \end{bmatrix}
\]

\begin{algorithm}[H]
  \caption {Flip de una matriz}
  \begin{algorithmic}
    \Statex \textbf{Define:} flipMatrix
    \Statex \textbf{Input:} \texttt{matrix} (matriz de \(n \times m\))
    \Statex \textbf{Output:} La matriz espejada horizontalmente

    \For{$i = 0$ \textbf{to} $n - 1$}
      \For{$j = 0$ \textbf{to} $m / 2 - 1$}
        \State $temp \gets matrix[i][j]$
        \State $matrix[i][j] \gets matrix[i][m - j - 1]$
        \State $matrix[i][m - j - 1] \gets temp$
      \EndFor
    \EndFor
  \end{algorithmic}
\end{algorithm}

\subsubsection{Tranposición de matrices}

La matriz tranpuesta $A^T$ de una matriz $A$ es una matriz que se obtiene de intercambiando las filas y las columnas de
la matriz original $A^T_{i, j} = A_{j, i}$.

\[
  \begin{bmatrix}
  a_{11} & a_{12} & a_{13} \\
  a_{21} & a_{22} & a_{23}
  \end{bmatrix}
  \xrightarrow{\text{tranposición}}
  \begin{bmatrix}
  a_{11} & a_{21} \\
  a_{12} & a_{22} \\
  a_{13} & a_{23}
  \end{bmatrix}
\]

\begin{algorithm}[H]
  \caption {Tranposición de una matriz}
  \begin{algorithmic}
    \Statex \textbf{Define:} transposeMatrix
    \Statex \textbf{Input:} \texttt{matrix} (matriz de \(n \times m\))
    \Statex \textbf{Output:} \texttt{output} (matriz tranpuesta de \(m \times n\))

    \For{$i = 0$ \textbf{to} $n - 1$}
      \For{$j = 0$ \textbf{to} $m - 1$}
        \State $output[j][i] \gets matrix[i][j]$
      \EndFor
    \EndFor
  \end{algorithmic}
\end{algorithm}

\newpage % ------------------------------------------------------------------------------------------------------------

\subsubsection{Rotación de matrices}

La rotación de una matriz es la operación que rota los elementos de la matriz en sentido horario o antihorario. Es
bastante sencillo de implementar utilizando una matriz de salida y recorriendo los elementos de la matriz de entrada de
la siguiente manera $A_{i, j} \rightarrow B_{j, n - i}$ para la rotación en sentido horario.

\[
  \begin{bmatrix}
  a_{11} & a_{12} & a_{13} \\
  a_{21} & a_{22} & a_{23}
  \end{bmatrix}
  \xrightarrow{\text{rotación en sentido horario}}
  \begin{bmatrix}
  a_{21} & a_{11} \\
  a_{22} & a_{12} \\
  a_{23} & a_{13}
  \end{bmatrix}
\]

Es interesante señalar que la rotación de una matriz cuadrada de \(n \times n\) en sentido horario se puede realizar
\textit{in-place} (sin utilizar memoria extra). Transponiendo la matriz y luego haciendo un flip horizontal de la
matriz, como se muestra en el siguiente algoritmo:

\begin{algorithm}[H]
  \caption {Rotación de una matriz cuadrada (in-place) en sentido horario}
  \begin{algorithmic}
    \Statex \textbf{Define:} rotateMatrix
    \Statex \textbf{Input:} \texttt{matrix} (matriz cuadrada de \(n \times n\))
    \Statex \textbf{Output:} La matriz rotada en sentido horario

    \For{$i = 0$ \textbf{to} $n - 1$}
      \For{$j = i + 1$ \textbf{to} $n - 1$}
        \State $temp \gets matrix[i][j]$
        \State $matrix[i][j] \gets matrix[j][i]$
        \State $matrix[j][i] \gets temp$
      \EndFor
    \EndFor

    \For{$i = 0$ \textbf{to} $n - 1$}
      \For{$j = 0$ \textbf{to} $n / 2 - 1$}
        \State $temp \gets matrix[i][j]$
        \State $matrix[i][j] \gets matrix[i][n - j - 1]$
        \State $matrix[i][n - j - 1] \gets temp$
      \EndFor
    \EndFor
  \end{algorithmic}
\end{algorithm}

\newpage % ------------------------------------------------------------------------------------------------------------

\subsection{Búsqueda y Ordenamiento}

Muchos algoritmos eficientes se basan en ordenar los datos de entrada antes de resolver el problema, dado que el
ordenamiento a menudo facilita la resolución del problema. En esta sección discutiremos algo de teoría de búsqueda y
ordenamiento de datos.

El problema básico de ordenamiento es el siguiente: Dado un \textit{array} que contiene \(n\) elementos, ordenarlos de
manera ascendente. Es bastante fácil resolver este problema en $O(n^2)$, sin embargo existen algunos algoritmos que
resuelven este problema de manera más eficiente.

\subsubsection{\textit{Bubble Sort}}

Este algoritmo lárgamente visto en todos los cursos de programación, es el más fácil de explicar por su simplicidad y
facilidad de implementación. El algoritmo consiste en recorrer el array y comparar cada elemento con el siguiente, si el
elemento actual es mayor que el siguiente, se intercambian los elementos. Este proceso se repite hasta que no se
realizan más intercambios, lo que significa que el array está ordenado.

\begin{lstlisting}[language=C]
for(int i = 0; i < n; i++) {
  for(int j = 0; j < n-1; j++) {
    if(arr[j] > arr[j + 1]) {
      // Intercambiar los elementos
      int temp = arr[j];
      arr[j] = arr[j + 1];
      arr[j + 1] = temp;
    }
  }
}
\end{lstlisting}

\textbf{Inversiones}: Un concepto importante en el ordenamiento de datos es el concepto de \textit{inversiones}. Una
inversión de los índices \texttt{(a, b)} tales que \(a < b\) y \texttt{arr[a] > arr[b]} (o lo que es lo mismo, que los
elementos están en el orden incorrecto).
El número de inversiones en un array es una medida de cuán desordenado está el array. Por ejemplo, si el array está
totalmente ordenado, el número de inversiones es \(0\), y si el array está en orden inverso, el número de inversiones
es:

\[
  1 + 2 + 3 + \ldots + (n - 1) = \frac{n(n - 1)}{2}
\]

Esto es porque intercambiar dos elementos del array reduce exactamente en \(1\) el número de inversiones. Con lo cual si
un algoritmo sólo puede reducir el número de inversiones en \(1\) por iteración, el número de iteraciones necesarias
para ordenar el array es al menos $O(n^2)$.

\subsubsection{\textit{Merge Sort}}

Si queremos crear un algoritmo de ordenamiento más eficiente, podemos utilizar el algoritmo \textit{Merge Sort}. Este
algoritmo nos permite ordenar elementos que están en partes diferentes del array. El algoritmo entra en el conjunto de
algoritmos que permiten ordenar un array en \(O(n \log n)\) tiempo.

El algoritmo merge sort intenta ordenar el subarray \texttt{arr[a..b]} de la siguiente manera:

\begin{enumerate}
  \item Si $a = b$, el array contiene un solo elemento por lo que se considera ordenado.
  \item Calculamos la posición del elemento del medio $\lfloor \frac{a + b}{2} \rfloor$.
  \item Recursivamente ordenamos el subarray \texttt{arr[a..k]}
  \item Recursivamente ordenamos el subarray \texttt{arr[k + 1..b]}
  \item Combinamos (\textit{merge}) el subarray \texttt{arr[a..k]} y el subarray \texttt{arr[k + 1..b]} en un solo array
    ordenado.
\end{enumerate}

El algoritmo \textit{merge sort} es eficiente porque divide el array en dos mitades cada vez y el merge de las mitades
se hace en tiempo lineal porque ya están ordenadas. Dado que hay $log(n)$ niveles de división y procesar cada nivel toma
$O(n)$ tiempo, el tiempo total de ejecución es \(O(n \log n)\).

\subsubsection{Bucket Sort / Counting Sort}

En vista de lo que vimos anteriormente con resepcto a las inversiones, podemos pensar si se puede ordenar un array en
menos complejidad que \(O(n log n)\). Y la respuesta es sí, pero para ello no tenemos que comparar elementos entre sí,
en caso contrario, no podemos superar la barrera de \(O(n log n)\).

El algoritmo \textit{Bucket Sort} es un algoritmo de ordenamiento que utiliza un array para contar la cantidad de veces
que aparece cada elemento del array. El algoritmo funciona de la siguiente manera:

\begin{enumerate}
  \item Creamos un array de tamaño \(k\) donde \(k\) es el número de elementos distintos del array a ordenar.
  \item Recorremos el array y contamos la cantidad de veces que aparece cada elemento, guardando el resultado en el array
    de conteo.
  \item Recorremos el array de conteo y escribimos los elementos en el array original en orden.
\end{enumerate}

La ventaja es que este algoritmo tiene una complejidad \(O(n)\), pero tiene la desventaja de que necesita un array de
tamaño \(k\) para contar los elementos.

\subsubsection{Conclusiones sobre el ordenamiento}

Aunque es importante entender los conceptos básicos de los algoritmos de ordenamiento, en la práctica suele ser una mala
idea implementar algoritmos de ordenamiento desde cero. Todos los lenguajes de programación tienen implementaciones de
ordenamiento eficientes y optimizadas. Por ejemplo, en C podemos utilizar la función \texttt{qsort} de la biblioteca
estándar, que implementa el algoritmo \textit{Quicker Sort} (un algoritmo de ordenamiento que es una variante del
algoritmo \textit{Quick Sort}). Pero tiene la desventaja de que no es estable, es decir, que no mantiene el orden de los
elementos iguales.

\newpage % ------------------------------------------------------------------------------------------------------------

\textbf{\textit{MergeSort} en C:} \\

\begin{lstlisting}[language=C]
#include <stdio.h>
#include <stdlib.h>

//
// MERGE function: Combina dos subarrays ordenados en uno solo
//
void merge(int array[], int left, int mid, int right) {
  // Calcula el size de los subarrays
  int leftSize = mid - left + 1;
  int rightSize = right - mid;

  // Crea arreglos temporales para los subarrays
  int* leftSubarray = (int*)malloc(leftSize * sizeof(int));
  int* rightSubarray = (int*)malloc(rightSize * sizeof(int));

  if (!leftSubarray || !rightSubarray) {
    fprintf(stderr, "No se pudo reservar la memoria\n");
    exit(1);
  }

  // Copiamos los datos a los subarrays temporales
  for (int i = 0; i < leftSize; i++)
    leftSubarray[i] = array[left + i];
  for (int j = 0; j < rightSize; j++)
    rightSubarray[j] = array[mid + 1 + j];

  // Combinamos los subarrays temporales en el array original
  int i = 0;    // Indice inicial del subarray izquierdo
  int j = 0;    // Indice inicial del subarray derecho
  int k = left; // Indice inicial del array combinado

  // Compar los elementos de los subarrays y combinarlos
  while (i < leftSize && j < rightSize) {
    if (leftSubarray[i] <= rightSubarray[j]) {
      array[k] = leftSubarray[i];
      i++;
    } else {
      array[k] = rightSubarray[j];
      j++;
    }
    k++;
  }

  // Copiar cualquier elemento restante del subarray izquierdo
  while (i < leftSize) {
    array[k] = leftSubarray[i];
    k++;
    i++;
  }

  // Copiar cualquier elemento restante del subarray derecho
  while (j < rightSize) {
    array[k] = rightSubarray[j];
    k++;
    j++;
  }

  // Liberar la memoria de los subarrays temporales
  free(leftSubarray);
  free(rightSubarray);
}

//
// mergeSort: implementa la estrategia divide-and-conquer para ordenar un array
//
void mergeSort(int array[], int left, int right) {
  if (left < right) {
    // Buscamos el punto medio del array (divide)
    int mid = left + (right - left) / 2;

    // Ordenamos recursivamente las dos mitades
    mergeSort(array, left, mid);        // Sort left half
    mergeSort(array, mid + 1, right);   // Sort right half

    // Combinamos las dos mitades ordenadas
    merge(array, left, mid, right);
  }
}

//
// Utility function to print an array
//
void printArray(int array[], int size) {
  for (int i = 0; i < size; i++)
    printf("%d ", array[i]);
  printf("\n");
}

//
// Main function to demonstrate merge sort
//
int main() {
  int arr[] = {38, 27, 43, 3, 9, 82, 10};
  int size = sizeof(arr) / sizeof(arr[0]);

  printf("Array original:\n");
  printArray(arr, size);

  mergeSort(arr, 0, size - 1);

  printf("\nArray Ordenado:\n");
  printArray(arr, size);

  return 0;
}
\end{lstlisting}

\newpage % ------------------------------------------------------------------------------------------------------------

\subsection{Problemas que se pueden resolver con ordenamiento}

\subsubsection{Encontrar duplicados en un array}

Por ejemplo supongamos que queremos saber si los elementos de un array son todos diferentes entre sí. Claramente el
algoritmo por fuerza bruta es el más sencillo de implementar, sin embargo ese algoritmo tiene una complejidad de
\(O(n^2)\).

\begin{lstlisting}[language=C]
bool unicos = true;
for (int i = 0; i < n; i++) {
  for (int j = i + 1; j < n; j++) {
    if (arr[i] == arr[j]) {
      unicos = false;
      break;
    }
  }
}
\end{lstlisting}

Sin embargo, si ordenamos el array primero, el problema puede ser resuelto en \(O(n \log n)\). En la práctica tendrás un
ejercicio para resolver este problema.

\subsubsection{Búsqueda binaria}

Supongamos que tenemos un array ordenado y queremos saber si un elemento está presente en el array. El algoritmo de
búsqueda binaria es un algoritmo eficiente que nos permite encontrar un elemento en un array ordenado en \(O(\log n)\)
tiempo. La implementación del algoritmo en C es la siguiente:

\begin{lstlisting}[language=C]
int busquedaBinaria(int arr[], int n, int x) {
  int left = 0, right = n - 1;
  while (left <= right) {
    int mid = (right + left) / 2; // Encontrar el punto medio
    if (arr[mid] == x) return mid; // Encontramos el elemento!!
    if (x > arr[mid]) left = mid + 1; // Buscar en la mitad derecha
    else right = mid - 1; // Buscar en la mitad izquierda
  }
  return -1; // No se encontro el elemento
}
\end{lstlisting}

\newpage % ------------------------------------------------------------------------------------------------------------

\subsection{Árboles}

Los grafos son una estructura de datos que consiste en un conjunto de nodos conectados por aristas, se utilizan para
representar relaciones entre datos. Los grafos pueden ser dirigidos o no dirigidos, y pueden tener ciclos o no. Un grafo
dirigido es aquel en el que las aristas tienen una dirección, es decir, que van de un nodo a otro nodo
específico, mientras que un grafo no dirigido es aquel en el que las aristas no tienen
dirección, es decir, que la relación entre los nodos es bidireccional.

Un árbol es un caso especial de grafo, que es \textbf{no dirigido}, \textbf{acíclico} y \textbf{conexo}. Es decir,
que un árbol es un grafo que no tiene ciclos y que está conectado (esto es, que existe un camino entre cualquier par de
nodos). Cada árbol tiene un nodo raíz, que es el nodo superior del árbol, y cada nodo puede tener cero o más nodos
hijos. Los árboles son una estructura de datos muy útil para representar jerarquías entre datos, como por ejemplo un
sistema de archivos o una base de datos.

Uno de los árboles más comunes es el árbol binario, que es un árbol en el que cada nodo tiene como máximo dos hijos, y
es una estructura de datos muy utilizada en algoritmos de búsqueda y ordenamiento.

Si bien los árboles se pueden representar de muchas maneras, la más común es utilizar una estructura de datos que
contenga un valor y dos punteros a los nodos hijos izquierdo y derecho. Por ejemplo:

\begin{lstlisting}[language=C]
struct Nodo {
    int valor;
    struct Nodo* izquierdo;
    struct Nodo* derecho;
};
\end{lstlisting}

Esto tiene la ventaja de que podemos representar árboles de cualquier tamaño y forma, y es fácil de implementar en C;
otra forma de representar un árbol es utilizando un array, donde el nodo en la posición \(i\) tiene como hijos los nodos
en las posiciones \(2i + 1\) y \(2i + 2\). Esta representación es útil para árboles completos o casi completos, ya que
si el árbol está muy desbalanceado, la representación en array puede desperdiciar mucha memoria.

\subsubsection{Recorridos de árboles}

Los recorridos de árboles son una forma de visitar todos los nodos de un árbol en un orden específico. Los tres
recorridos más comunes son el recorrido en preorden, el recorrido en inorden y el recorrido en postorden.

\begin{itemize}
  \item \textbf{Preorden:} En este recorrido, se visita primero el nodo raíz, luego se recorre el subárbol izquierdo y
    finalmente se recorre el subárbol derecho. El orden de visita es: raíz, izquierda, derecha.
  \item \textbf{Inorden:} En este recorrido, se recorre primero el subárbol izquierdo, luego se visita el nodo raíz y
    finalmente se recorre el subárbol derecho. El orden de visita es: izquierda, raíz, derecha.
  \item \textbf{Postorden:} En este recorrido, se recorre primero el subárbol izquierdo, luego se recorre el subárbol
    derecho y finalmente se visita el nodo raíz. El orden de visita es: izquierda, derecha, raíz.
\end{itemize}

\newpage % ------------------------------------------------------------------------------------------------------------

Supongamos que tenemos el siguiente árbol binario:

\[
      1
     / \
    2   3
   / \   \
  4   5   6
\]

El recorrido en preorden sería: 1, 2, 4, 5, 3, 6 \\

\begin{lstlisting}[language=C]
#include <stdio.h>
#include <stdlib.h>

typedef struct Node {
    int value;
    struct Node* left;
    struct Node* right;
} Node;

Node* createNode(int value) {
    Node* newNode = (Node*)malloc(sizeof(Node));
    if (!newNode) {
        perror("Error al asignar memoria");
        exit(EXIT_FAILURE);
    }
    newNode->value = value;
    newNode->left = NULL;
    newNode->right = NULL;
    return newNode;
}

void preorder(Node* root) {
    if (root == NULL) return;
    printf("%d ", root->value); // Visitar nodo
    preorder(root->left);       // Recorrer subárbol izquierdo
    preorder(root->right);      // Recorrer subárbol derecho
}

int main() {
    // Construcción del árbol:
    //
    //       1
    //      / \
    //     2   3
    //    / \   \
    //   4   5   6
    //
    Node* root = createNode(1);
    root->left = createNode(2);
    root->right = createNode(3);
    root->left->left = createNode(4);
    root->left->right = createNode(5);
    root->right->right = createNode(6);

    printf("Recorrido en preorden: ");
    preorder(root);
    printf("\n");

    return 0;
}
\end{lstlisting}



\input{../shared.tex/common_footers.tex}

\end{document}
