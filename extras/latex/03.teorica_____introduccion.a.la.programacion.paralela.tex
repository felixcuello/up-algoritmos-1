% \documentclass[10pt, twocolumn, a4paper]{article}
\documentclass[12pt,a4paper]{article}

\usepackage[utf8]{inputenc}                                             % To handle UTF-8 input
\usepackage[T1]{fontenc}                                                % Better font encoding that supports accents
\usepackage[backend=biber, style=ieee]{biblatex}                        % To include the bibliography
\usepackage[left=2cm, right=2cm, top=2.5cm, bottom=4cm]{geometry}     % To set the margins
\usepackage[noend]{algpseudocode}
\usepackage[table]{xcolor}                                              % For coloring cells

\usepackage{algorithm}                                                  % To include algorithms
\usepackage{amsfonts}                                                   % To include math fonts:ToggleTerm direction=float
\usepackage{amsmath}                                                    % To include Mathematic symbols
\usepackage{authblk}                                                    % To format author affiliations
\usepackage{caption}                                                    % For caption spacing
\usepackage{enumitem}                                                   % To customize lists (items like i, ii, iii, iv)
\usepackage{float}                                                      % To place figures where you want them
\usepackage{fancyhdr}                                                   % To customize headers and footers
\usepackage{graphicx}                                                   % To include images
\usepackage{hyperref}                                                   % To include hyperlinks
\usepackage{lipsum}                                                     % TODO: remove this
\usepackage{pgfplots}                                                   % To plot functions
\usepackage{listings}                                                   % To include code
\usepackage{tabularx}                                                   % For equal-width columns
\usepackage{tcolorbox}                                                  % To make colored boxes
\usepackage{tikz}                                                       % To draw graphs
\usepackage{titlesec}                                                   % To format section titles
\usepackage{xcolor}                                                     % To define colors

\usetikzlibrary{graphs,graphs.standard, arrows.meta}
\usetikzlibrary{positioning}

\addbibresource{./references.bib}

%%%%%%%%%%%%%%%%%%%%%%%%%%%%%%%%%%%%%%%%%%%%%%%%%%%%%%%%%%%%%%%%%%%%%%%%%%%%%%%%
% Estilo de código
%%%%%%%%%%%%%%%%%%%%%%%%%%%%%%%%%%%%%%%%%%%%%%%%%%%%%%%%%%%%%%%%%%%%%%%%%%%%%%%%
\definecolor{up-blue}{RGB}{0,102,254}
\definecolor{up-violet}{RGB}{100,30,180}
\definecolor{up-gray}{RGB}{102,102,102}
\definecolor{up-red}{RGB}{152,102,52}
\definecolor{up-comment}{RGB}{153,153,153}
\definecolor{code-bg}{RGB}{248,248,248}
\definecolor{code-border}{RGB}{104,104,104}

\lstdefinestyle{up-code}{
    backgroundcolor=\color{code-bg},
    commentstyle=\color{up-comment}\itshape\footnotesize,
    keywordstyle=\color{up-blue}\bfseries,
    numberstyle=\tiny\color{up-red},
    stringstyle=\color{up-violet},
    basicstyle=\ttfamily\footnotesize,
    breakatwhitespace=false,
    breaklines=true,
    captionpos=b,
    keepspaces=true,
    numbers=left,
    numbersep=5pt,
    showspaces=false,
    showstringspaces=false,
    showtabs=false,
    tabsize=2,
    frame=single,
    frameround=tttt,
    rulecolor=\color{code-border},
    xleftmargin=5pt,
    xrightmargin=5pt,
    upquote=true,
    columns=fixed,
    extendedchars=true,
    inputencoding=utf8,
    literate=
        {á}{{\'a}}1 {é}{{\'e}}1 {í}{{\'i}}1 {ó}{{\'o}}1 {ú}{{\'u}}1
        {Á}{{\'A}}1 {É}{{\'E}}1 {Í}{{\'I}}1 {Ó}{{\'O}}1 {Ú}{{\'U}}1
        {ñ}{{\~n}}1 {Ñ}{{\~N}}1
        {ü}{{\"u}}1 {Ü}{{\"U}}1
        {¿}{{?`}}1 {¡}{{!`}}1
}

\lstset{style=up-code}
%%%%%%%%%%%%%%%%%%%%%%%%%%%%%%%%%%%%%%%%%%%%%%%%%%%%%%%%%%%%%%%%%%%%%%%%%%%%%%%%


% Para poder hacer flechas
\usetikzlibrary{shapes, arrows}

% Sección de definiciones
\titleformat{\section}{\Large\bfseries}{\thesection}{1em}{}
\titleformat{\subsection}{\large\bfseries}{\thesubsection}{1em}{}

% Caja de colores
\definecolor{mint}{RGB}{202,251,202}
\definecolor{yellow}{RGB}{255,255,202}
\definecolor{red}{RGB}{255,202,202}

% Variables globales para el documento
\newcommand{\universityname}{Universidad de Palermo}
\newcommand{\facultyname}{Facultad de Ingeniería}
\newcommand{\coursename}{\textbf{Algoritmos 1}}
\newcommand{\currentsemester}{Primer Semestre}
\newcommand{\currentyear}{2026}
\newcommand{\copyrightnotice}{
  \scriptsize \textcopyright{} \currentyear \space \universityname. Prohibida la reproducción total o parcial de imágenes y textos
}

% Esto es para poder agregar comentarios al código
\newcommand{\comentario}[1]{\textcolor{gray}{// #1}}

% Definir los encabezados y pies de página
\pagestyle{fancy}
\fancyhf{} % Borra encabezados y pies de página

% Configuración del header
\fancyhead[R]{\includegraphics[height=50px]{../latex/images/logo_up.jpg}}
\renewcommand{\headrulewidth}{0.4pt} % Agregar línea debajo del encabezado
\fancyhead[L]{\scriptsize \facultyname}
\fancyhead[C]{\small \coursename}
\setlength{\headheight}{50pt} % Ajustar la altura del encabezado

% Configuración del footer
\fancyfoot[C]{\copyrightnotice}
\fancyfoot[R]{\thepage}
\renewcommand{\footrulewidth}{0.4pt} % Agregar línea arriba del pie de página
\setlength{\footskip}{40pt} % Agregar separación entre el texto y el pie de página

% Configuración de la primera página
\fancypagestyle{firstpage}{
  \fancyhf{} % Borra encabezados y pies de página
  \fancyfoot[C]{\thepage} % Número de página centrado
  \renewcommand{\headrulewidth}{0pt} % Sin línea en el encabezado
  \renewcommand{\footrulewidth}{0.4pt} % Sin línea en el pie de página
}

% Configuración de la primera página
\AtBeginDocument{\thispagestyle{firstpage}}


\begin{document}

\begin{center}
  \LARGE\textbf{\coursename} \\
  \Large{Teórica 03 - Introducción a la Programación Paralela} \\
  \normalsize{\currentsemester, \currentyear} \\
  \vspace{1em}
  \hrule
\end{center}

\vspace{1em}

\setcounter{section}{3}

\newpage

\tableofcontents

\newpage

\subsection{Introducción}
\label{sec:introduccion}

Hasta ahora hemos visto cómo escribir en C para resolver problemas algorítmicos utilizando la CPU de una computadora.
Sin embargo, el objetivo de la materia es la de aprender a escribir programas que puedan ejecutarse en múltiples
unidades de procesamiento en forma \textbf{paralela}.

Los microprocesadores que tenemos en nuestras computadoras personales y celulares se basan en una unidad central de
procesamiento (CPU) que ejecuta un cierto número de \textit{threads} (hilos) en paralelo, que ejecutan un código
secuencial de instrucciones. A lo largo de la historia, estas CPUs se fueron llevando a límites de rendimiento cada vez
mayores, donde gracias a la miniaturización de los componentes, la mayor cantidad de núcleos, la mayor velocidad del
reloj, mejores formas de enfriamiento y la mejora en la eficiencia energética, se logró aumentar la cantidad de
procesamiento que se podía hacer en un solo chip.

Esto produjo que, la mayor parte de las aplicaciones, se vieran beneficiadas de estos avances de hardware para
incrementar la velocidad de las propias aplicaciones donde, esencialmente, el mismo software funcionaba mucho más rápido
a medida que se iban realizando mejoras en estas unidades de procesamiento secuenciales. Sin embargo esto muchas veces
se lo conoce como \href{https://es.wikipedia.org/wiki/Escalabilidad#Escalabilidad_vertical}{escalabilidad vertical} ,
donde se busca mejorar el rendimiento simplemente agregando más recursos.

El problema de la escalabilidad vertical es que tiene un límite, ya que eventualmente llegaremos a un punto donde no se
pueden acelerar más las unidades de procesamiento y debemos encontrar otras formas de optimización. Es aquí donde entra
en juego la \textbf{escalabilidad horizontal} que se refiere a la capacidad de aumentar el rendimiento de un sistema
agregando más unidades de procesamiento (CPUs o GPUs) al sistema en lugar de hacer las unidades más veloces.

Para ilustrar los conceptos básicos de la programación paralela y escalable, necesitamos programar en un lenguaje que
soporte la paralelización masiva como C, C++ o Python. Para ello elegimos uno de los modelos de programación paralela
más populares que es CUDA (\textit{Compute Unified Device Architecture}) que es una extensión del lenguaje C para
ilustrar los conceptos básicos de la programación paralela. El modelo de programación CUDA fue desarrollado por
\href{https://www.nvidia.com/es-la/}{NVIDIA} que permite aprovechar la potencia de cómputo de varias unidades (GPU) de
procesamiento a la vez.

La idea de parlelizar no es nueva, pero históricamente los centros de procesamiento paralelo estaban limitados a
supercomputadoras y clusters bajo la órbita de gobiernos y grandes empresas que podían costearlos
(\textcite{sutter2005}). Esta paralelización NO implica necesariamente un reemplazo de las CPUs sino un complemento ya
que por un lado las CPUs son buenas para tareas de baja latencia y su poder de procesamiento seguramente seguirá
aumentando.

El \textit{ratio} de rendimiento entre una CPU y una GPU puede ser de 1:10 (o más) para ciertas operaciones. Las CPUs
están optimizadas para ejecutar código secuencial de forma \textit{performante} relizando, en la medida de lo posible,
un paralelismo interno de instrucciones que es transparente para el usuario. Las CPUs, además, poseen internamente
grandes cachés que les permiten manejar la inherente latencia del acceso a dispositivos externos lentos, mientras que
las GPUs por el otro lado tienen cachés mucho más pequeñas ya que sólo necesitan acceder a memoria (muy rápida) y tener
una gran optimización para mover grandes cantidades de datos tanto \textit{in} como \textit{out} de la DRAM
(\textit{Dynamic Random Access Memory}) ya que originalmente fueron diseñadas para procesar gráficos en tiempo real,
especialmente en la industria de los videojuegos donde el
\href{https://www.youtube.com/shorts/8wj3zVA03WQ}{\textit{frame buffering}} es un requerimiento crítico.

\subsection{Complejidad algorítmica}

En el repaso de algoritmos (y materias anteriores esta) vieron sólo algoritmos secuenciales. Sin embargo antes de
avanzar con la programación paralela, tenemos que estudiar \textit{complejidad algorítmica} más en detalle.

La complejidad algorítmica es un parámetro con el cual podemos medir la eficiencia de un algoritmo en
términos de la cantidad de recursos que va a utilizar para completar la tarea. Por un lado tenemos la complejidad en
\textbf{tiempo}, que podemos pensarla como la cantidad de ciclos de CPU que necesita un algoritmo para obtener el
resultado esperado, y por el otro lado, tenemos la complejidad en \textbf{espacio}, que podemos pensarlo como cantidad
de memoria que necesita un algoritmo para obtener el resultado esperado.

Ambas complejidades se expresan en función del tamaño del \textit{input} del algoritmo y nos permiten comparar
diferentes algoritmos independientemente del lenguaje en que estén implementados. Esto significa que si tenemos dos
lenguajes de programación diferentes, la complejidad tanto en \textbf{tiempo} como en \textbf{espacio} de un mismo
algoritmo será la misma, aunque el tiempo de ejecución y la cantidad de memoria pueden ser diferentes ya que los
lenguajes de programación tienen diferentes niveles de optimización.

\subsubsection{Complejidad en Tiempo}

Cuando diseñamos un algoritmo, es importante tener en cuenta la cantidad de recursos que se van a utilizar, tratando de
estimar cuál será su complejidad en tiempo y en espacio de acuerdo a la solución que hayamos propuesto. Hay diferentes
métricas de estimación, aunque las más comunes son estimar \textbf{el caso promedio} y, aún más importante, \textbf{el
peor caso posible}. Por ejemplo, agregar un elemento al final de un \textit{array} dinámico en cualquier lenguaje de
programación de alto nivel, tiene una \textbf{complejidad promedio} constante, ya que cuando se crea un array dinámico,
el lenguaje lo crea de un tamaño dado (aún si el usuario no lo especifica). Por el otro lado, si el \textit{array}
dinámico estuviera lleno, estaríamos en el \textit{peor caso} posible, ya que el lenguaje, tendría que conseguir más
memoria para añadir ese nuevo elemento y luego copiar todos los elementos a la nueva posición de memoria.

En esta materia vamos a ver sólo la notación de complejidad para el peor caso, que se representa con la notación
\textbf{Big O}. En palabras simples, lo que representa la complejidad algorítmica en el peor caso es: \textit{la
cantidad máxima de tiempo de CPU o cantidad máxima de memoria que se requiere para resolver un problema en función del
tamaño de la entrada}. Esta función \textbf{Big O} nos da una cota asintótica superior de la cantidad de recursos que se
necesitan para resolver un problema en función del tamaño de la entrada.

En la figura \ref{fig:big_o_comparison}, podemos observar diferentes funciones de complejidad algorítmica en función del
tamaño de la entrada (\texttt{n}).

\begin{figure}[H]
  \centering
  \includegraphics[width=200px]{./images/big_o_comparison.png}
  \caption{Big O - Comparación de complejidades}
  \label{fig:big_o_comparison}
\end{figure}

Diferentes implementaciones de un mismo algoritmo para resolver un problema pueden tener diferentes complejidades. Por
ejemplo, existen varios algoritmos que resuelven el problema de ordenar un \textit{array} (arreglo) de elementos de
tamaño $n$. El más popular por su facilidad de implementación es el algoritmo de
\href{https://es.wikipedia.org/wiki/Ordenamiento_de_burbuja}{burbujeo} que, si bien ordena los elementos, tiene una
complejidad de $O(n^2)$, pero si utilizamos un algoritmo de ordenamiento más eficiente, nos encontramos con el
\href{https://es.wikipedia.org/wiki/Ordenamiento_por_mezcla}{\textit{merge sort}} que tiene una complejidad de $O(n
\cdot \; \log \; n)$. Ambos algoritmos resuelven el mismo problema (ordernar un array), pero uno es mucho más eficiente
que el otro. En la figura \ref{fig:bubble_sort_vs_quick_sort} podemos ver la comparación de ambas complejidades. \\

\textbf{Descubrir un algoritmo más eficiente que otro no fácil.}

\begin{figure}[H]
  \centering
  \includegraphics[width=300px]{./images/bubble_sort_vs_quick_sort.png}
  \caption{Big O - Comparación de complejidad algorítmica para Bubble Sort vs Quick Sort}
  \label{fig:bubble_sort_vs_quick_sort}
\end{figure}

\subsubsection{Definición: Big O}

En todos los gráficos que vimos la variable \texttt{n} es la cantidad de elementos que tenemos como \textit{input} de nuestro
algoritmo que se ejecutará con una cierta función de tiempo $f(n)$.

\textbf{Definición}: Diremos que el algoritmo se comporta con una complejidad asintótica de $O(g(n))$ si existe una
constante $C$ y un valor $n_0$ tal que el valor absoluto de $f(n)$ es menor o igual a $C$ por el valor absoluto de
$g(n)$ para todo $n$ mayor a $n_0$. (\textcite{wilf2002}). Formalmente:

\[
  f(n) = O(g(n)) \; (n \; \rightarrow \; \infty) \; si \; \exists \; C, n_0 \; \slash \; \lvert f(n) \rvert \leq C \,
  \lvert g(n) \rvert \; (\forall \; n > n_0)
\]

Las complejidades algorítmicas nos dan una idea de si un algoritmo puede ser realizable o no por una determinada unidad
de procesamiento; por \textit{realizable} queremos decir que la capacidad de resolver las instancias deseadas de un
problema estén dentro de nuestros recursos disponibles. En la práctica, la factibilidad es muy dependiente del contexto
y no es particularmente \textit{portable} (portátil) entre diferentes problemas y situaciones. Un principio común se
mantiene en casi todas las situaciones: \textbf{una tasa de crecimiento exponencial en el consumo de algún recurso
limita la aplicación del uso de ese método a todas las instancias, excepto a las más pequeñas}. En otras palabras, si
nuestros algoritmos tienden a tener complejidades exponenciales (o más) en el tamaño de la entrada, no importa cuántos
recursos tengamos, no podremos resolver instancias grandes del problema.

Por lo tanto, la factibilidad ha llegado a significar que la tasa de crecimiento del recurso está acotada por un una
complejidad polinomial en el tamaño de la entrada. Esto nos da la noción común de que un problema tiene una solución
secuencial factible sólo si tenemos un algoritmo de tiempo polinomial, es decir, sólo si cualquier instancia de tamaño
$n$ del problema se puede resolver en tiempo $n^{O(1)}$. Aunque ampliamente reconocido como muy simplista, la dicotomía
entre algoritmos polinomiales y no polinomiales ha demostrado ser un discriminador poderoso entre aquellos cálculos que
son factibles en la práctica y aquellos que no lo son (\textcite{greenlaw1995}).

Lógicamente hay problemas para los cuales no se han encontrado algoritmos polinomiales, por ejemplo, la factorización
entera de un número muy grande es un problema que no puede resolverse en tiempo polinomial con las CPUs y GPUs actuales.
Por eso es que la seguridad de muchos sistemas de encriptación se basan en ello. Si bien excede lárgamente el contenido
de esta materia, hay algoritmos cuánticos que pueden resolver este problema en tiempo $O((\log n)^{O(1)})$
(\cite{ekert1996}).

\subsubsection{Complejidad en Espacio}

La complejidad en espacio es la cantidad de memoria requerida para resolver un algoritmo. Es, también, una función que
depende de la entrada del algoritmo y de la cantidad de memoria adicional que se necesite para resolver el problema.

Como en la complejidad en tiempo, se utiliza la notación \textbf{Big O} ya que en este caso también se trata de una cota
asintótica superior de la cantidad de memoria que se necesita para resolver un problema en función del tamaño de la
entrada. Esto incluye tanto la memoria utilizada estáticamente como la memoria utilizada dinámicamente, sin embargo, no
se tiene en cuenta la memoria utilizada como \textit{input}. Por ejemplo, un algoritmo que cuenta la cantidad de números
pares de un vector tiene una complejidad en tiempo de \textbf{$O(n)$} y una complejidad en espacio de \textbf{$O(1)$} ya
que sólo necesita una variable para contar la cantidad de números pares.

\begin{algorithm}
\caption{Contar números pares}
  \label{alg:countPares}
  \begin{algorithmic}[1]
    \Statex \textbf{Define:} countPares($a$)
    \Statex \textbf{Input:} $a = \{a_0, a_1, a_2, \ldots, a_n\}$
    \Statex \textbf{Initialization:} $count = 0$
    \State \textbf{Output:} $count$

    \For{$i = 0$ to $n$}
      \If{$a_i \mod 2 = 0$}
        \State $count = count + 1$
      \EndIf
    \EndFor
  \end{algorithmic}
\end{algorithm}

Vemos aquí que la complejidad en tiempo es $O(n)$ ya que sólo hay un ciclo que recorre todos los elementos del vector
de tamaño $n$ pero la complejidad en espacio es $O(1)$ ya que sólo utilizamos una variable \texttt{count} para guardar
el contador con la cantidad de números pares. Esto significa que si el tamaño del vector $a$ se duplica, la cantidad de
memoria utilizada no cambia, pero la cantidad de ciclos de CPU necesarios para completar la tarea se duplica.

\subsection{¿Por qué no se utilizan GPUs para todo?}

En la \ref{sec:introduccion} se mencionó que el \textit{ratio} de rendimiento entre una CPU y una GPU puede ser de 1:10
(e incluso mayor), entonces una pregunta válida sería \textbf{¿por qué no se utilizan GPUs para todo?}. La respuesta a
esto no es única, por un lado las GPUs están diseñadas para tener alto \textit{throughput} de operaciones matemáticas y
de transferencia a memoria y suelen ser buenas para tareas de alta latencia que requieren pasajes grandes de información
desde y hacia la memoria (óptimos para el procesamiento paralelo). Por el otro lado las CPUs están optimizadas para
realizar tareas de baja latencia y tareas inherentemente secuenciales donde sorbepasan holgadamente a la
\textit{performance} de una GPU. Por esto es que estos sistemas se llaman híbridos, ya que hay un hilo conductor
secuencial y otros que se ejecutan en paralelo (a demanda).

Un ejemplo típico de una tarea que es inherentemente secuencial es la ejecución de un sistema operativo, ya que
todas las tareas que se ejecutan en un sistema operativo son secuenciales y dependen unas de otras. Otroe ejemplo es la
orquestación GPU-CPU, ya que la CPU es la que orquesta las tareas que se ejecutan en la GPU y no al revés. Además la CPU
es buenísima para tareas de baja latencia como la ejecución del software de una computadora personal.

Por el otro lado, las GPUs son muy buenas para tareas que pueden ser paralelizadas y que requieren un alto
\textit{throughput} de operaciones matemáticas y de transferencia a memoria. Un ejemplo típico de una tarea que puede ser
paralelizada es el procesamiento de imágenes, ya que cada píxel de una imagen puede ser procesado de manera
independiente. Otro ejemplo es el entrenamiento de un modelo de machine learning implica realizar millones o incluso
miles de millones de operaciones matriciales (multiplicaciones de matrices) y cálculos matemáticos para ajustar los
parámetros del modelo.

Al desarrollar CUDA C, NVIDIA no sólo ha permitido paralelizar aplicaicones, sino que también ha permitido bajar el
costo del desarrollo del software paralelo que estaba restringido a supercomputadoras masivamente paralelas. Esto se
debe a que el mercado actual de GPUs es enorme ya que casi todas las PCs actuales tienen algún tipo de GPU instalada,
habiendo más de 1000 millones de GPUs en el mundo que pueden ser utilizadas con CUDA, NVIDIA facilitó el desarrollo de
aplicaciones paralelas, logrando que el desarrollo de software paralelo sea más accesible para todos.

\subsection{Ley de Amdahl}

Como veremos en la sección siguiente, no todas las aplicaciones pueden ser paralelizadas, y en muchas aplicaciones, sólo
un porcentage del tiempo de ejecución de una aplicación puede ser paralelizado. El algoritmo utilizado es quien define
la mejora de velocidad y no necesariamente el número de procesadores; en algún momento llegaremos a un límite donde
finalmente por más que tengamos más recursos no se podrá paralelizar más el algoritmo. A esto se lo conoce como
\href{https://es.wikipedia.org/wiki/Ley_de_Amdahl}{ley de Amdahl}.

La ley de Amdahl es una fórmula aritmética simple que se utiliza para estimar la potencial mejora de velocidad que puede
obtenerse al paralelizar una parte del código de un programa secuencial en múltiples procesadores. La idea de la fórmula
es que hay una parte que puede ser paralelizada y otra parte que no puede ser paralelizada, y la mejora de velocidad se
calcula en función del porcentaje de código que se puede paralelizar y el número de procesadores utilizados.
\cite{Tuomanen2018HandsOnGPU}

La ley de Amdahl se expresa matemáticamente como:

\[
  S = \frac{1}{(1 - P) + \frac{P}{N}}
\]

donde:

\begin{itemize}
  \item $S$ es la velocidad de mejora del programa.
  \item $P$ es la fracción del programa que se puede paralelizar (entre 0 y 1).
  \item $N$ es el número de procesadores utilizados.
\end{itemize}

Es bastante sencillo ver que si la parte paralelizable es $0\%$ (es decir, $P = 0$), no existe mejora de velocidad
($S = 1$). Por otro lado, si la parte paralelizable es $100\%$ (es decir, $P = 1$), la mejora de velocidad es igual al
número de procesadores utilizados ($S = N$).

Supongamos que $P = 0.5$ (es decir, el $50\%$ del programa puede ser paralelizado) y que utilizamos $N = 2$ (es decir,
utilizamos dos procesadores). En este caso, la mejora de velocidad sería:

\[
  S = \frac{1}{(1 - 0.5) + \frac{0.5}{2}} = \frac{1}{0.5 + 0.25} = \frac{1}{0.75} \approx 1.33
\]

\subsection{Desafíos en la programación paralela}

Podríamos pensar entonces que la programación paralela puede ser una solución general para resolver todos los problemas
de \textit{performance} y debería utilizarse siempre. Aunque como vimos con la ley de Amdahl, no todos los problemas son
$100\%$ paralelizables, y en general sólo podremos intentar paralelizar una parte de la aplicación, mientras que otra
parte deberá seguir siendo secuencial.

El primer problema, es que es difícil diseñar algoritmos paralelos, ya que requiere pensar los problemas de formas
anti-intuitivas que, muchas veces, no son sencillas de resolver para que tengan el mismo nivel de complejidad
computacional que los algoritmos secuenciales. La performance de los algoritmos paralelos son muy sensibles a los datos
de entrada, ya que esencialmente la paralelización se basa en el procesamiento de datos de manera limitada por la
velocidad de acceso a memoria. Pero aún si salváramos estos problemas que son, méramente técnicos, nos encontraríamos
con la pregunta más profunda: \textbf{¿todos los problemas son teóricamente paralelizables?}.

\newpage % ------------------------------------------------------------------------------------------------------------

\subsection{Los límites de la programación paralela}

En la teoría de complejidad computacional, P es una clase \footnote{En matemáticas, una clase es una colección de
conjuntos (u otros objetos matemáticos) que pueden ser definidos unívocamente por una propiedad compartida por todos los
miembros de ese conjunto} de complejidad que contiene todos los problemas de decisión que pueden ser resueltos por una
máquina de Turing determinista en tiempo polinomial. Sin embargo,
\href{https://en.wikipedia.org/wiki/Nick_Pippenger}{Nicholas Pippenger} realizó una investigación exhaustiva sobre
circuitos con profundidad polilogarítmica y tamaño polinomial y sugirió una interesante clase de complejidad a estudiar
que sería la clase de problemas solucionables por máquinas paralelas en tiempo polilogarítmico \footnote{Un problema es
polilogarítmico si su complejidad es $O((\log n)^k)$ para algún $k$} usando un número polinomial de procesadores. Esta
clase se conoce como \textbf{NC} (por \textit{Nick's class}). (\textcite{stockmeyer1987})

Por otro lado los problemas $P-completos$ son de interés porque parecen carecer de soluciones altamente paralelas.
Es decir, los diseñadores de algoritmos han fallado en encontrar algoritmos \textit{NC} para cualquier problema
$P-completo$. En consecuencia, la promesa de la computación paralela, es decir, que aplicar más procesadores a un
problema parece ser imposible en toda la clase de problemas $P-completos$. Dejando abierta la siguiente pregunta:
(\textcite{greenlaw1995})

\[
  P \stackrel{?}{=} NC
\]

Esta clase P se puede pensar como los problemas tratables (tesis de Cobham), por lo que \textit{NC} se puede pensar como
los problemas que se pueden resolver eficientemente en una computadora paralela. NC es un subconjunto de P porque los
cálculos paralelos polilogarítmicos se pueden simular mediante cálculos secuenciales polinomiales. Pero no se sabe si
$NC \stackrel{?}{=} P$, pero \textbf{la mayoría de los investigadores sospechan que esto es falso}, lo que significa que
probablemente hay algunos problemas tratables que son \textbf{"intrínsecamente secuenciales"} y no se pueden acelerar
significativamente utilizando paralelismo. Así como la clase NP-completo se puede pensar como "probablemente
intratable", así que \textbf{la clase P-completo, cuando se utilizan reducciones NC, se puede pensar como "probablemente
no paralelizable" o "probablemente \textit{intrínsecamente secuencial}"}. Pero por más que parezca extraño, no hay
ninguna demostración formal de que $P \neq NC$ o que $P = NC$. \\

Con lo cual, la respuesta a la pregunta de si todos los problemas son paralelizables es que \textbf{NO, no todos los
problemas son paralelizables ya que hay ciertos problemas que son intrínsecamente secuenciales} (o no se ha encontrado
una forma de paralelizarlos).

\subsection{Modelos y Lenguajes de Programación Paralela}

Hay varios lenguajes de programación paralela propuestos para abordar el problema de la programación paralela, aunque
aquí sólo veremos CUDA, aquí tienen una lista de algunos de los lenguajes más populares:

\begin{itemize}
  \item \textbf{OpenMP}: es una API de programación en paralelo que se basa en directivas de compilador y funciones de
    biblioteca para permitir la paralelización de aplicaciones en sistemas de memoria compartida.
  \item \textbf{MPI}: es una biblioteca de paso de mensajes que permite la comunicación entre procesos en un sistema
    distribuido.
  \item \textbf{OpenACC}: Es un estándar de programación para la computación paralela desarrollado por Cray, CAPS,
    Nvidia y PGI. El estándar está diseñado para simplificar la programación paralela de sistemas heterogéneos CPU/GPU.
  \item \textbf{OpenCL}: Es un estándar abierto para la programación de sistemas heterogéneos que consta de CPUs, GPUs y
    otros dispositivos de cómputo.
  \item \textbf{CUDA}: Es una plataforma de computación paralela y un modelo de programación desarrollado por NVIDIA para
    sus GPUs.
\end{itemize}

\input{../shared.tex/common_footers.tex}

\end{document}
