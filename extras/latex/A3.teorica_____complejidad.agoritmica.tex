% \documentclass[10pt, twocolumn, a4paper]{article}
\documentclass[12pt,a4paper]{article}

\usepackage[utf8]{inputenc}                                             % To handle UTF-8 input
\usepackage[T1]{fontenc}                                                % Better font encoding that supports accents
\usepackage[backend=biber, style=ieee]{biblatex}                        % To include the bibliography
\usepackage[left=2cm, right=2cm, top=2.5cm, bottom=4cm]{geometry}     % To set the margins
\usepackage[noend]{algpseudocode}
\usepackage[table]{xcolor}                                              % For coloring cells

\usepackage{algorithm}                                                  % To include algorithms
\usepackage{amsfonts}                                                   % To include math fonts:ToggleTerm direction=float
\usepackage{amsmath}                                                    % To include Mathematic symbols
\usepackage{authblk}                                                    % To format author affiliations
\usepackage{caption}                                                    % For caption spacing
\usepackage{enumitem}                                                   % To customize lists (items like i, ii, iii, iv)
\usepackage{float}                                                      % To place figures where you want them
\usepackage{fancyhdr}                                                   % To customize headers and footers
\usepackage{graphicx}                                                   % To include images
\usepackage{hyperref}                                                   % To include hyperlinks
\usepackage{lipsum}                                                     % TODO: remove this
\usepackage{pgfplots}                                                   % To plot functions
\usepackage{listings}                                                   % To include code
\usepackage{tabularx}                                                   % For equal-width columns
\usepackage{tcolorbox}                                                  % To make colored boxes
\usepackage{tikz}                                                       % To draw graphs
\usepackage{titlesec}                                                   % To format section titles
\usepackage{xcolor}                                                     % To define colors

\usetikzlibrary{graphs,graphs.standard, arrows.meta}
\usetikzlibrary{positioning}

\addbibresource{./references.bib}

%%%%%%%%%%%%%%%%%%%%%%%%%%%%%%%%%%%%%%%%%%%%%%%%%%%%%%%%%%%%%%%%%%%%%%%%%%%%%%%%
% Estilo de código
%%%%%%%%%%%%%%%%%%%%%%%%%%%%%%%%%%%%%%%%%%%%%%%%%%%%%%%%%%%%%%%%%%%%%%%%%%%%%%%%
\definecolor{up-blue}{RGB}{0,102,254}
\definecolor{up-violet}{RGB}{100,30,180}
\definecolor{up-gray}{RGB}{102,102,102}
\definecolor{up-red}{RGB}{152,102,52}
\definecolor{up-comment}{RGB}{153,153,153}
\definecolor{code-bg}{RGB}{248,248,248}
\definecolor{code-border}{RGB}{104,104,104}

\lstdefinestyle{up-code}{
    backgroundcolor=\color{code-bg},
    commentstyle=\color{up-comment}\itshape\footnotesize,
    keywordstyle=\color{up-blue}\bfseries,
    numberstyle=\tiny\color{up-red},
    stringstyle=\color{up-violet},
    basicstyle=\ttfamily\footnotesize,
    breakatwhitespace=false,
    breaklines=true,
    captionpos=b,
    keepspaces=true,
    numbers=left,
    numbersep=5pt,
    showspaces=false,
    showstringspaces=false,
    showtabs=false,
    tabsize=2,
    frame=single,
    frameround=tttt,
    rulecolor=\color{code-border},
    xleftmargin=5pt,
    xrightmargin=5pt,
    upquote=true,
    columns=fixed,
    extendedchars=true,
    inputencoding=utf8,
    literate=
        {á}{{\'a}}1 {é}{{\'e}}1 {í}{{\'i}}1 {ó}{{\'o}}1 {ú}{{\'u}}1
        {Á}{{\'A}}1 {É}{{\'E}}1 {Í}{{\'I}}1 {Ó}{{\'O}}1 {Ú}{{\'U}}1
        {ñ}{{\~n}}1 {Ñ}{{\~N}}1
        {ü}{{\"u}}1 {Ü}{{\"U}}1
        {¿}{{?`}}1 {¡}{{!`}}1
}

\lstset{style=up-code}
%%%%%%%%%%%%%%%%%%%%%%%%%%%%%%%%%%%%%%%%%%%%%%%%%%%%%%%%%%%%%%%%%%%%%%%%%%%%%%%%


% Para poder hacer flechas
\usetikzlibrary{shapes, arrows}

% Sección de definiciones
\titleformat{\section}{\Large\bfseries}{\thesection}{1em}{}
\titleformat{\subsection}{\large\bfseries}{\thesubsection}{1em}{}

% Caja de colores
\definecolor{mint}{RGB}{202,251,202}
\definecolor{yellow}{RGB}{255,255,202}
\definecolor{red}{RGB}{255,202,202}

% Variables globales para el documento
\newcommand{\universityname}{Universidad de Palermo}
\newcommand{\facultyname}{Facultad de Ingeniería}
\newcommand{\coursename}{\textbf{Algoritmos 1}}
\newcommand{\currentsemester}{Primer Semestre}
\newcommand{\currentyear}{2026}
\newcommand{\copyrightnotice}{
  \scriptsize \textcopyright{} \currentyear \space \universityname. Prohibida la reproducción total o parcial de imágenes y textos
}

% Esto es para poder agregar comentarios al código
\newcommand{\comentario}[1]{\textcolor{gray}{// #1}}

% Definir los encabezados y pies de página
\pagestyle{fancy}
\fancyhf{} % Borra encabezados y pies de página

% Configuración del header
\fancyhead[R]{\includegraphics[height=50px]{../latex/images/logo_up.jpg}}
\renewcommand{\headrulewidth}{0.4pt} % Agregar línea debajo del encabezado
\fancyhead[L]{\scriptsize \facultyname}
\fancyhead[C]{\small \coursename}
\setlength{\headheight}{50pt} % Ajustar la altura del encabezado

% Configuración del footer
\fancyfoot[C]{\copyrightnotice}
\fancyfoot[R]{\thepage}
\renewcommand{\footrulewidth}{0.4pt} % Agregar línea arriba del pie de página
\setlength{\footskip}{40pt} % Agregar separación entre el texto y el pie de página

% Configuración de la primera página
\fancypagestyle{firstpage}{
  \fancyhf{} % Borra encabezados y pies de página
  \fancyfoot[C]{\thepage} % Número de página centrado
  \renewcommand{\headrulewidth}{0pt} % Sin línea en el encabezado
  \renewcommand{\footrulewidth}{0.4pt} % Sin línea en el pie de página
}

% Configuración de la primera página
\AtBeginDocument{\thispagestyle{firstpage}}


\begin{document}

\begin{center}
  \LARGE\textbf{\coursename} \\
  \Large{Apéndice 3 - Complejidad Algorítmica} \\
  \normalsize{\currentsemester, \currentyear} \\
  \vspace{1em}
  \hrule
\end{center}

\vspace{1em}

\setcounter{section}{3}

\newpage

\tableofcontents

\newpage

\subsection{Introducción}
\label{sec:introduccion}

Los microprocesadores que tenemos en nuestras computadoras personales y celulares se basan en una unidad central de
procesamiento (CPU) que ejecuta un cierto número de \textit{threads} (hilos) en paralelo, que ejecutan un código
secuencial de instrucciones. A lo largo de la historia, estas CPUs se fueron llevando a límites de rendimiento cada vez
mayores, donde gracias a la miniaturización de los componentes, la mayor cantidad de núcleos, la mayor velocidad del
reloj, mejores formas de enfriamiento y la mejora en la eficiencia energética, se logró aumentar la cantidad de
procesamiento que se podía hacer en un solo chip.

La idea del análisis de complejidad algorítmica es pensar en la cantidad de recursos que se necesitan para resolver un
problema, y cómo esta cantidad de recursos se relaciona con el tamaño de la entrada del problema. Esto es un componente
fundamental del diseño y la implementación de cualquier algoritmo, ya que el análisis de complejidad algorítmica nos
permite entender qué algoritmo es más eficiente para resolver un problema dado, y cómo se comportará el algoritmo a
medida que el tamaño de la entrada crezca.

\subsection{Complejidad algorítmica}

En el repaso de algoritmos (y materias anteriores esta) vieron sólo algoritmos secuenciales. Sin embargo antes de
avanzar con la programación paralela, tenemos que estudiar \textit{complejidad algorítmica} más en detalle.

La complejidad algorítmica es un parámetro con el cual podemos medir la eficiencia de un algoritmo en
términos de la cantidad de recursos que va a utilizar para completar la tarea. Por un lado tenemos la complejidad en
\textbf{tiempo}, que podemos pensarla como la cantidad de ciclos de CPU que necesita un algoritmo para obtener el
resultado esperado, y por el otro lado, tenemos la complejidad en \textbf{espacio}, que podemos pensarlo como cantidad
de memoria que necesita un algoritmo para obtener el resultado esperado.

Ambas complejidades se expresan en función del tamaño del \textit{input} del algoritmo y nos permiten comparar
diferentes algoritmos independientemente del lenguaje en que estén implementados. Esto significa que si tenemos dos
lenguajes de programación diferentes, la complejidad tanto en \textbf{tiempo} como en \textbf{espacio} de un mismo
algoritmo será la misma, aunque el tiempo de ejecución y la cantidad de memoria pueden ser diferentes ya que los
lenguajes de programación tienen diferentes niveles de optimización.

\subsubsection{Complejidad en Tiempo}

Cuando diseñamos un algoritmo, es importante tener en cuenta la cantidad de recursos que se van a utilizar, tratando de
estimar cuál será su complejidad en tiempo y en espacio de acuerdo a la solución que hayamos propuesto. Hay diferentes
métricas de estimación, aunque las más comunes son estimar \textbf{el caso promedio} y, aún más importante, \textbf{el
peor caso posible}. Por ejemplo, agregar un elemento al final de un \textit{array} dinámico en cualquier lenguaje de
programación de alto nivel, tiene una \textbf{complejidad promedio} constante, ya que cuando se crea un array dinámico,
el lenguaje lo crea de un tamaño dado (aún si el usuario no lo especifica). Por el otro lado, si el \textit{array}
dinámico estuviera lleno, estaríamos en el \textit{peor caso} posible, ya que el lenguaje, tendría que conseguir más
memoria para añadir ese nuevo elemento y luego copiar todos los elementos a la nueva posición de memoria.

En esta materia vamos a ver sólo la notación de complejidad para el peor caso, que se representa con la notación
\textbf{Big O}. En palabras simples, lo que representa la complejidad algorítmica en el peor caso es: \textit{la
cantidad máxima de tiempo de CPU o cantidad máxima de memoria que se requiere para resolver un problema en función del
tamaño de la entrada}. Esta función \textbf{Big O} nos da una cota asintótica superior de la cantidad de recursos que se
necesitan para resolver un problema en función del tamaño de la entrada.

En la figura \ref{fig:big_o_comparison}, podemos observar diferentes funciones de complejidad algorítmica en función del
tamaño de la entrada (\texttt{n}).

\begin{figure}[H]
  \centering
  \includegraphics[width=200px]{./images/big_o_comparison.png}
  \caption{Big O - Comparación de complejidades}
  \label{fig:big_o_comparison}
\end{figure}

Diferentes implementaciones de un mismo algoritmo para resolver un problema pueden tener diferentes complejidades. Por
ejemplo, existen varios algoritmos que resuelven el problema de ordenar un \textit{array} (arreglo) de elementos de
tamaño $n$. El más popular por su facilidad de implementación es el algoritmo de
\href{https://es.wikipedia.org/wiki/Ordenamiento_de_burbuja}{burbujeo} que, si bien ordena los elementos, tiene una
complejidad de $O(n^2)$, pero si utilizamos un algoritmo de ordenamiento más eficiente, nos encontramos con el
\href{https://es.wikipedia.org/wiki/Ordenamiento_por_mezcla}{\textit{merge sort}} que tiene una complejidad de $O(n
\cdot \; \log \; n)$. Ambos algoritmos resuelven el mismo problema (ordernar un array), pero uno es mucho más eficiente
que el otro. En la figura \ref{fig:bubble_sort_vs_quick_sort} podemos ver la comparación de ambas complejidades. \\

\textbf{Descubrir un algoritmo más eficiente que otro no fácil.}

\begin{figure}[H]
  \centering
  \includegraphics[width=300px]{./images/bubble_sort_vs_quick_sort.png}
  \caption{Big O - Comparación de complejidad algorítmica para Bubble Sort vs Quick Sort}
  \label{fig:bubble_sort_vs_quick_sort}
\end{figure}

\subsubsection{Definición: Big O}

En todos los gráficos que vimos la variable \texttt{n} es la cantidad de elementos que tenemos como \textit{input} de nuestro
algoritmo que se ejecutará con una cierta función de tiempo $f(n)$.

\textbf{Definición}: Diremos que el algoritmo se comporta con una complejidad asintótica de $O(g(n))$ si existe una
constante $C$ y un valor $n_0$ tal que el valor absoluto de $f(n)$ es menor o igual a $C$ por el valor absoluto de
$g(n)$ para todo $n$ mayor a $n_0$. (\textcite{wilf2002}). Formalmente:

\[
  f(n) = O(g(n)) \; (n \; \rightarrow \; \infty) \; si \; \exists \; C, n_0 \; \slash \; \lvert f(n) \rvert \leq C \,
  \lvert g(n) \rvert \; (\forall \; n > n_0)
\]

Las complejidades algorítmicas nos dan una idea de si un algoritmo puede ser realizable o no por una determinada unidad
de procesamiento; por \textit{realizable} queremos decir que la capacidad de resolver las instancias deseadas de un
problema estén dentro de nuestros recursos disponibles. En la práctica, la factibilidad es muy dependiente del contexto
y no es particularmente \textit{portable} (portátil) entre diferentes problemas y situaciones. Un principio común se
mantiene en casi todas las situaciones: \textbf{una tasa de crecimiento exponencial en el consumo de algún recurso
limita la aplicación del uso de ese método a todas las instancias, excepto a las más pequeñas}. En otras palabras, si
nuestros algoritmos tienden a tener complejidades exponenciales (o más) en el tamaño de la entrada, no importa cuántos
recursos tengamos, no podremos resolver instancias grandes del problema.

Por lo tanto, la factibilidad ha llegado a significar que la tasa de crecimiento del recurso está acotada por un una
complejidad polinomial en el tamaño de la entrada. Esto nos da la noción común de que un problema tiene una solución
secuencial factible sólo si tenemos un algoritmo de tiempo polinomial, es decir, sólo si cualquier instancia de tamaño
$n$ del problema se puede resolver en tiempo $n^{O(1)}$. Aunque ampliamente reconocido como muy simplista, la dicotomía
entre algoritmos polinomiales y no polinomiales ha demostrado ser un discriminador poderoso entre aquellos cálculos que
son factibles en la práctica y aquellos que no lo son (\textcite{greenlaw1995}).

Lógicamente hay problemas para los cuales no se han encontrado algoritmos polinomiales, por ejemplo, la factorización
entera de un número muy grande es un problema que no puede resolverse en tiempo polinomial con las CPUs y GPUs actuales.
Por eso es que la seguridad de muchos sistemas de encriptación se basan en ello. Si bien excede lárgamente el contenido
de esta materia, hay algoritmos cuánticos que pueden resolver este problema en tiempo $O((\log n)^{O(1)})$
(\cite{ekert1996}).

\subsubsection{Complejidad en Espacio}

La complejidad en espacio es la cantidad de memoria requerida para resolver un algoritmo. Es, también, una función que
depende de la entrada del algoritmo y de la cantidad de memoria adicional que se necesite para resolver el problema.

Como en la complejidad en tiempo, se utiliza la notación \textbf{Big O} ya que en este caso también se trata de una cota
asintótica superior de la cantidad de memoria que se necesita para resolver un problema en función del tamaño de la
entrada. Esto incluye tanto la memoria utilizada estáticamente como la memoria utilizada dinámicamente, sin embargo, no
se tiene en cuenta la memoria utilizada como \textit{input}. Por ejemplo, un algoritmo que cuenta la cantidad de números
pares de un vector tiene una complejidad en tiempo de \textbf{$O(n)$} y una complejidad en espacio de \textbf{$O(1)$} ya
que sólo necesita una variable para contar la cantidad de números pares.

\begin{algorithm}
\caption{Contar números pares}
  \label{alg:countPares}
  \begin{algorithmic}[1]
    \Statex \textbf{Define:} countPares($a$)
    \Statex \textbf{Input:} $a = \{a_0, a_1, a_2, \ldots, a_n\}$
    \Statex \textbf{Initialization:} $count = 0$
    \State \textbf{Output:} $count$

    \For{$i = 0$ to $n$}
      \If{$a_i \mod 2 = 0$}
        \State $count = count + 1$
      \EndIf
    \EndFor
  \end{algorithmic}
\end{algorithm}

Vemos aquí que la complejidad en tiempo es $O(n)$ ya que sólo hay un ciclo que recorre todos los elementos del vector
de tamaño $n$ pero la complejidad en espacio es $O(1)$ ya que sólo utilizamos una variable \texttt{count} para guardar
el contador con la cantidad de números pares. Esto significa que si el tamaño del vector $a$ se duplica, la cantidad de
memoria utilizada no cambia, pero la cantidad de ciclos de CPU necesarios para completar la tarea se duplica.

\end{document}
