% \documentclass[10pt, twocolumn, a4paper]{article}
\documentclass[12pt,a4paper]{article}

\usepackage[utf8]{inputenc}                                             % To handle UTF-8 input
\usepackage[T1]{fontenc}                                                % Better font encoding that supports accents
\usepackage[backend=biber, style=ieee]{biblatex}                        % To include the bibliography
\usepackage[left=2cm, right=2cm, top=2.5cm, bottom=4cm]{geometry}     % To set the margins
\usepackage[noend]{algpseudocode}
\usepackage[table]{xcolor}                                              % For coloring cells

\usepackage{algorithm}                                                  % To include algorithms
\usepackage{amsfonts}                                                   % To include math fonts:ToggleTerm direction=float
\usepackage{amsmath}                                                    % To include Mathematic symbols
\usepackage{authblk}                                                    % To format author affiliations
\usepackage{caption}                                                    % For caption spacing
\usepackage{enumitem}                                                   % To customize lists (items like i, ii, iii, iv)
\usepackage{float}                                                      % To place figures where you want them
\usepackage{fancyhdr}                                                   % To customize headers and footers
\usepackage{graphicx}                                                   % To include images
\usepackage{hyperref}                                                   % To include hyperlinks
\usepackage{lipsum}                                                     % TODO: remove this
\usepackage{pgfplots}                                                   % To plot functions
\usepackage{listings}                                                   % To include code
\usepackage{tabularx}                                                   % For equal-width columns
\usepackage{tcolorbox}                                                  % To make colored boxes
\usepackage{tikz}                                                       % To draw graphs
\usepackage{titlesec}                                                   % To format section titles
\usepackage{xcolor}                                                     % To define colors

\usetikzlibrary{graphs,graphs.standard, arrows.meta}
\usetikzlibrary{positioning}

\addbibresource{./references.bib}

%%%%%%%%%%%%%%%%%%%%%%%%%%%%%%%%%%%%%%%%%%%%%%%%%%%%%%%%%%%%%%%%%%%%%%%%%%%%%%%%
% Estilo de código
%%%%%%%%%%%%%%%%%%%%%%%%%%%%%%%%%%%%%%%%%%%%%%%%%%%%%%%%%%%%%%%%%%%%%%%%%%%%%%%%
\definecolor{up-blue}{RGB}{0,102,254}
\definecolor{up-violet}{RGB}{100,30,180}
\definecolor{up-gray}{RGB}{102,102,102}
\definecolor{up-red}{RGB}{152,102,52}
\definecolor{up-comment}{RGB}{153,153,153}
\definecolor{code-bg}{RGB}{248,248,248}
\definecolor{code-border}{RGB}{104,104,104}

\lstdefinestyle{up-code}{
    backgroundcolor=\color{code-bg},
    commentstyle=\color{up-comment}\itshape\footnotesize,
    keywordstyle=\color{up-blue}\bfseries,
    numberstyle=\tiny\color{up-red},
    stringstyle=\color{up-violet},
    basicstyle=\ttfamily\footnotesize,
    breakatwhitespace=false,
    breaklines=true,
    captionpos=b,
    keepspaces=true,
    numbers=left,
    numbersep=5pt,
    showspaces=false,
    showstringspaces=false,
    showtabs=false,
    tabsize=2,
    frame=single,
    frameround=tttt,
    rulecolor=\color{code-border},
    xleftmargin=5pt,
    xrightmargin=5pt,
    upquote=true,
    columns=fixed,
    extendedchars=true,
    inputencoding=utf8,
    literate=
        {á}{{\'a}}1 {é}{{\'e}}1 {í}{{\'i}}1 {ó}{{\'o}}1 {ú}{{\'u}}1
        {Á}{{\'A}}1 {É}{{\'E}}1 {Í}{{\'I}}1 {Ó}{{\'O}}1 {Ú}{{\'U}}1
        {ñ}{{\~n}}1 {Ñ}{{\~N}}1
        {ü}{{\"u}}1 {Ü}{{\"U}}1
        {¿}{{?`}}1 {¡}{{!`}}1
}

\lstset{style=up-code}
%%%%%%%%%%%%%%%%%%%%%%%%%%%%%%%%%%%%%%%%%%%%%%%%%%%%%%%%%%%%%%%%%%%%%%%%%%%%%%%%


% Para poder hacer flechas
\usetikzlibrary{shapes, arrows}

% Sección de definiciones
\titleformat{\section}{\Large\bfseries}{\thesection}{1em}{}
\titleformat{\subsection}{\large\bfseries}{\thesubsection}{1em}{}

% Caja de colores
\definecolor{mint}{RGB}{202,251,202}
\definecolor{yellow}{RGB}{255,255,202}
\definecolor{red}{RGB}{255,202,202}

% Variables globales para el documento
\newcommand{\universityname}{Universidad de Palermo}
\newcommand{\facultyname}{Facultad de Ingeniería}
\newcommand{\coursename}{\textbf{Algoritmos 1}}
\newcommand{\currentsemester}{Primer Semestre}
\newcommand{\currentyear}{2026}
\newcommand{\copyrightnotice}{
  \scriptsize \textcopyright{} \currentyear \space \universityname. Prohibida la reproducción total o parcial de imágenes y textos
}

% Esto es para poder agregar comentarios al código
\newcommand{\comentario}[1]{\textcolor{gray}{// #1}}

% Definir los encabezados y pies de página
\pagestyle{fancy}
\fancyhf{} % Borra encabezados y pies de página

% Configuración del header
\fancyhead[R]{\includegraphics[height=50px]{../latex/images/logo_up.jpg}}
\renewcommand{\headrulewidth}{0.4pt} % Agregar línea debajo del encabezado
\fancyhead[L]{\scriptsize \facultyname}
\fancyhead[C]{\small \coursename}
\setlength{\headheight}{50pt} % Ajustar la altura del encabezado

% Configuración del footer
\fancyfoot[C]{\copyrightnotice}
\fancyfoot[R]{\thepage}
\renewcommand{\footrulewidth}{0.4pt} % Agregar línea arriba del pie de página
\setlength{\footskip}{40pt} % Agregar separación entre el texto y el pie de página

% Configuración de la primera página
\fancypagestyle{firstpage}{
  \fancyhf{} % Borra encabezados y pies de página
  \fancyfoot[C]{\thepage} % Número de página centrado
  \renewcommand{\headrulewidth}{0pt} % Sin línea en el encabezado
  \renewcommand{\footrulewidth}{0.4pt} % Sin línea en el pie de página
}

% Configuración de la primera página
\AtBeginDocument{\thispagestyle{firstpage}}


\begin{document}

\begin{center}
  \LARGE\textbf{\coursename} \\
  \Large{Introducción al Lenguaje C++} \\
  \normalsize{\currentsemester, \currentyear} \\
  \vspace{1em}
  \hrule
\end{center}

\vspace{1em}

\setcounter{section}{1}

\newpage

\tableofcontents

\newpage

\begin{tcolorbox}[colback=mint,colframe=mint!75!black,arc=0pt,outer arc=0pt]
  \textbf{¡Bienvenidos a Programación con Algoritmos 1!} \\

  Esta es una guía adicional para que tengas de referencia del lenguaje C++ que vamos a utilizar en la materia. El
  objetivo de esta guía es que puedas:

  \begin{itemize}
    \item \textbf{Compilar y ejecutar código en C / C++}
    \item \textbf{Aprender y repasar los conceptos básicos de C / C++} que son necesarios para la materia.
  \end{itemize}

  ¡Consultá todo lo que no entiendas de la práctica!
\end{tcolorbox}

\newpage

\subsection{Introducción}
\label{sec:set_up}

Para poder trabajar en la materia vas a tener que tener instalado un compilador C++ para poder probar tus propios
ejercicios y problemas. Si sólo vas a correr los ejemplos y problemas vistos en clase podés utilizar simplemente la
plataforma de programación competitiva de la universidad en:
\href{https://cloud-stack.palermo.edu:35500/users/sign_in}{UP Competitive Programming Platform}.

\subsection{¿Cómo instalo un compilador de C++?}

La forma más sencilla es utilizando \texttt{docker}. \href{https://www.docker.com/products/docker-desktop/}{Docker
Desktop} es una aplicación que permite crear y administrar contenedores de Docker en tu computadora. Un contenedor de
Docker es una instancia de una imagen de docker (muy similar a una máquina virtual) que se ejecuta de manera aislada del
sistema operativo subyacente, lo que permite ejecutar aplicaciones y servicios de manera consistente en diferentes
entornos.

La ventaja de utilizar docker es que no estás instalando nada más que docker (que es muy utilizada en entornos
profesionales, con lo cual seguramente lo sigas usando), y luego todo lo que se instala dentro del contenedor es
totalmente efímero y si un día querés eliminarlo, simplemente eliminás el contenedor y listo, sin tener que preocuparte
por desinstalar nada ni por conflictos de versiones de software.

\subsubsection{Crear el contenedor}
\label{sec:crear_contenedor}

La primera cosa que tenés que hacer es crear el contenedor de docker con el compilador que vamos a utilizar en la
materia. Para eso tenés que ir al directorio \texttt{docker/} que se encuentra en este repositorio y ejecutar:

\begin{verbatim}
docker build -t algoritmos_1 .
\end{verbatim}

Esto va a crear una imagen de docker con el nombre \texttt{algoritmos\_1} que tiene instalado el compilador de C++ y
todas las herramientas necesarias para poder compilar y ejecutar código en C++.

\textbf{¡Con esto ya podés correr los ejemplos y ejercicios de la materia!}

\newpage

\subsection{Hello World en C++}

En general uno asocia C++ con la programación orientada a objetos, sin embargo en esta materia, vamos a utilizar el
lenguaje C++ como un C potenciado. Es decir, vamos a utilizar la sintaxis de C++ pero sin utilizar las
características de la programación orientada a objetos, pero vamos a tomar ventaja de las bibliotecas de alto nivel que
ofrece, como las estructuras de datos ya incluídas y sus algoritmos para poder tener un código que se ejecute en alto
nivel que, además, pueda compilar C que es algo que verán en el resto de la carrera.

El código de ejemplo para imprimir un Hello World es un poco diferente en C++ que en C, porque la interfaz de entrada
y salida de datos es diferente. En C++ se utiliza la biblioteca \texttt{iostream} para la entrada y salida:

\begin{lstlisting}[language=C++, caption={Código de ejemplo Hello World en C++}]
#include <iostream>  // esto es lo mismo que en C, pero sin la extensión .h

int main() {
  // std::cout es el objeto de salida estándar en C++, y std::endl es un salto de línea
  std::cout << "Hola mundo.... desde C++!!!" << std::endl;
  return 0;
}
\end{lstlisting}

En general el estudio de Algoritmos se puede hacer en cualquier lenguaje, pero C es un lenguaje de muy bajo nivel que no
tiene en su biblioteca estándar algunas funcionalidades que necesitamos para poder estudiar cierto tipo de algoritmos.

\newpage

\subsection{Corriendo programas en C++}

Para ejecutar programas en C++ pueden utilizar una instalación local, pueden usar `docker` como ya vimos en la sección
anterior \ref{sec:crear_contenedor}, pueden utilizar la
\href{https://cloud-stack.palermo.edu:35500/users/sign_in}{plataforma de programación competitiva de la universidad}, o
pueden utilizar alguna plataforma como \href{https://onecompiler.com/cpp}{OneCompiler} o
\href{https://www.onlinegdb.com/online_c++_compiler}{OnlineGDB}.

En cualquier caso si van a utilizar un contenedor de docker o una instalación local de \texttt{g++} para compilar el
código, el comando para compilar un programa en C++ es el siguiente:

\begin{verbatim}
g++ -Wall -o programa programa.cc
\end{verbatim}

Donde \texttt{-Wall} es un flag que le indica al compilador que muestre todas las advertencias, \texttt{-o programa} es
el flag que le indica al compilador que el archivo ejecutable se llame \texttt{programa}, y \texttt{programa.cc} es el
archivo de código fuente que queremos compilar.

\newpage

\subsection{Repaso de C y C++}

\begin{tcolorbox}[colback=yellow,colframe=yellow!75!black,arc=0pt,outer arc=0pt]
\textbf{ATENCIÓN}

En esta sección vamos a repasar los conceptos básicos de C++, algunas cosas no van a estar explicadas porque el
  objetivo de esta materia no es ense\~nar C++, con lo cual si alguna de las cosas que están escritas acá no se
  entiende, podrás preguntar en clase o investigar un poco.

  Los conceptos IMPORTANTES van a estar explicados, pero la sintaxis del lenguaje no va a estar explicada en detalle,
  porque el libro de C++ en sí mismo es un libro de más de 600 páginas y no es obligatorio para la materia,
  simplemente aprendé la sintaxis.

  Es cierto que escribir \texttt{print "hola"} puede resultar más ameno que escribir \texttt{std::cout << "hola" <<
  std::endl}, pero en definitiva es una cuestión exclusivamente de sintaxis.

  Igual iremos explicando lo que significa cada parte del código, pero lo importante será enfocarnos en algoritmos.
\end{tcolorbox}

\subsubsection{Compilación y ejecución}

Este apartado está repetido en la práctica para que puedas ver cómo se compila y ejecuta un programa en C y C++.
Supongamos que tenés el siguiente código en un archivo llamado \texttt{cuadrado.cc}:

\textbf{Versión en C}

\begin{lstlisting}[language=C++, caption={Cuadrado de un número}]
#include <stdio.h>

int main() {
  int n;

  scanf("%d", &n); // Lee un número desde la entrada estándar
  printf("%d\n", n*n); // Imprime el cuadrado del número

  return 0;
}
\end{lstlisting}

\textbf{Versión en C++}

\begin{lstlisting}[language=C++, caption={Cuadrado de un número}]
#include <iostream>

using namespace std; // Esto no es obligatorio en versiones nuevas de C++

int main() {
  int n;

  cin >> n; // Lee un número desde la entrada estándar (similar a scanf en C)
  cout << n*n << endl; // Imprime el cuadrado del número (similar a printf en C)

  return 0;
}
\end{lstlisting}

\begin{verbatim}
g++ -Wall -o cuadrado cuadrado.cc
./cuadrado
5
25
\end{verbatim}

La idea es que cada vez que ejecutemos este programa, se quedará esperando a que ingresemos un número por pantalla y
luego imprimirá el cuadrado de ese número.


\subsubsection{Sintaxis básica}

El lenguaje C++ es un lenguaje de programación de propósito general, que se caracteriza por su simplicidad y
eficiencia. Un programa en C++ se compone de una o más funciones, siendo la función \texttt{main} la función principal
que se ejecuta al iniciar el programa. La sintaxis básica es la siguiente:

\begin{lstlisting}[language=C++, caption={Sintáxis básica de un programa en C++}]
// #include sirve para incluir bibliotecas
#include <iostream>  // Biblioteca para entrada y salida de datos
#include <algorithm> // Biblioteca para algoritmos predefinidos
#include <vector>    // Biblioteca para arrays dinámicos (vectores)
// etc.

// main es la funcion principal del programa que DEBE estar presente
// argc y argv son los argumentos de la linea de comandos
int main(int argc, char *argv[]) {
  cout << "Hola mundo.... desde C++!!!" << endl; // Imprime un mensaje en la consola
  return 0;
}
\end{lstlisting}


\subsubsection{Variables y tipos de datos}

En C++, las variables se utilizan para almacenar datos y deben ser declaradas antes de ser utilizadas. Los tipos de
datos más comunes en C son:

\begin{itemize}
  \item \texttt{int}: Enteros (números enteros).
  \item \texttt{float}: Números de punto flotante (decimales).
  \item \texttt{double}: Números de punto flotante de doble precisión.
  \item \texttt{char}: Caracteres individuales.
  \item \texttt{void}: Tipo de dato vacío, utilizado para funciones que no retornan un valor.
\end{itemize}

Además se pueden utilizar modificadores de tipo como \texttt{unsigned} para enteros sin signo, o \texttt{long} para
enteros de mayor tamaño. Por ejemplo, \texttt{unsigned int} es un entero sin signo que puede almacenar valores desde 0
hasta $2^{32-1}$, mientras que \texttt{long int} es un entero de mayor tamaño que puede almacenar valores desde
$-2^{63}$ hasta $2^{63-1}$.

\subsubsection{Operadores}

C cuenta con una variedad de operadores que se utilizan para realizar operaciones aritméticas, lógicas y de comparación.
Los operadores más comunes son:

\begin{itemize}
  \item Aritméticos: \texttt{+}, \texttt{-}, \texttt{*}, \texttt{/}, \texttt{\%}.
  \item Logicos: \texttt{\&\&} (AND), \texttt{\textbar\textbar} (OR), \texttt{!} (NOT).
  \item De comparacion: \texttt{==}, \texttt{!=}, \texttt{<}, \texttt{>}, \texttt{<=}, \texttt{>=}.
  \item Asignación: \texttt{=}, \texttt{+=}, \texttt{-=}, \texttt{*=}, \texttt{/=}, \texttt{\%=}.
  \item Bit a bit: \texttt{\&}, \texttt{\textbar}, \texttt{\textasciicircum}, \texttt{\textasciitilde}, \texttt{<<}, \texttt{>>}.
\end{itemize}

Los operadores aritméticos no requieren una explicación adicional, con excepción del operador \texttt{\%} que es el
operador de módulo que nos devuelve el resto de una división entera. Por ejemplo, si tenemos \texttt{5 \% 2} el
resultado es \texttt{1} porque \texttt{5 / 2} da como resultado \texttt{2} y el resto es \texttt{1}.

Los operadores lógicos se utilizan para combinar expresiones que devuelve un valor de verdad (\texttt{true} o
\texttt{false}. \textbf{En C y C++}: \texttt{NULL}, \texttt{0} o \texttt{false}, todo lo demás se considera
\texttt{true}.

Los operadores de comparación se utilizan para comparar dos valores y devolver un valor de verdad. Con lo cual si
tenemos una expresión como \texttt{5 > 3} el resultado es \texttt{true}, mientras que si tenemos \texttt{5 < 3} el
resultado es \texttt{false} y estas expresiones se pueden combinar con los operadores lógicos para formar expresiones
más complejas, por ejemplo \texttt{(a > 3) \&\& (b < 5)}.

Los operadores de asignación se utilizan para asignar un valor a una variable, por ejemplo \texttt{x = 5} asigna el
valor \texttt{5} a la variable \texttt{x}. Además hay operadores de asignación compuestos (\texttt{+=}, \texttt{-=},
\texttt{*=}, \texttt{/=}, \texttt{\%=}) son una forma abreviada de escribir una operación de asignación combinada con
una operación aritmética. Por ejemplo, \texttt{x += 5} es equivalente a \texttt{x = x + 5}.

En una operación de asignación existe el \texttt{lvalue} y el \texttt{rvalue}, donde el \texttt{lvalue} es la variable
donde se va a almacenar el resultado de la operación y el \texttt{rvalue} es el valor que se va a asignar. Esta es la
razón por la cual no se puede hacer una asignación como \texttt{5 = x} ya que \texttt{5} no es un \texttt{lvalue}, lo
que significa que no se puede asignar un valor a él.

Además existen operadores bit a bit se utilizan para realizar operaciones a nivel de bits, estos son importantes para
poder manipular datos a nivel de bits, lo cual es útil para la optimización de memoria.

\begin{itemize}
  \item \texttt{\&}: \texttt{5 \& 3} da como resultado \texttt{1} porque en binario \texttt{5} es \texttt{101} y
    \texttt{3} es \texttt{011}, y al hacer la operación AND bit a bit, el resultado es \texttt{001}.
  \item \texttt{\textbar}: \texttt{5 \textbar 3} da como resultado \texttt{7} porque en binario \texttt{5} es
    \texttt{101} y \texttt{3} es \texttt{011}, y al hacer la operación OR bit a bit, el resultado es \texttt{111}.
  \item \texttt{\textasciicircum}: \texttt{5 \textasciicircum 3} da como resultado \texttt{6} porque en binario
    \texttt{5} es \texttt{101} y \texttt{3} es \texttt{011}, y al hacer la operación XOR bit a bit, el resultado es
    \texttt{110}.
  \item \texttt{\textasciitilde}: \texttt{\textasciitilde} es el operador de complemento a uno, que invierte todos los
    bits de un número. Por ejemplo, \texttt{\textasciitilde 5} da como resultado \texttt{-6}.
  \item \texttt{<<}: \texttt{5 << 1} desplaza los bits de \texttt{5} una posición a la izquierda, lo que equivale a
    multiplicar por 2, dando como resultado \texttt{10}.
  \item \texttt{>>}: \texttt{5 >> 1} desplaza los bits de \texttt{5} una posición a la derecha, lo que equivale a
    la división entera por 2, dando como resultado \texttt{2}.
\end{itemize}


\subsubsection{Estructuras de control}

C cuenta con estructuras de control que permiten tomar decisiones y repetir bloques de código. Las estructuras más
comunes son:

\begin{itemize}
  \item Condicionales: \texttt{if}, \texttt{else if}, \texttt{else}
  \item Bucles: \texttt{for}, \texttt{while}
  \item Saltos: \texttt{break}, \texttt{continue}, \texttt{return}.
\end{itemize}

Ejemplos:

\begin{lstlisting}[language=C++, caption={Estructuras de Control}]
#include <stdio.h>

int main() {
  // Bucle for
  for (int x = -10; x < 11; x++) {
    if((x > 5) && (x % 3 == 0)) {
      printf("x=%d es mayor a 5 y multiplo de 3\n", x);
    }

    // Condicional if
    if (x > 0) {
      printf("x=%d es positivo\n", x);
    } else if (x < 0) {
      printf("x=%d es negativo\n", x);
    } else {
      printf("x=%d es cero\n", x);
    }
  }

  // Bucle while
  int j = 0;
  while (j < 5) {
    printf("Iteracion %d\n", j);
    j++;
  }

  return 0;
}
\end{lstlisting}


\subsubsection{Punteros}

Los punteros son una característica poderosa de C y C++ que permite manipular direcciones de memoria directamente. Un
puntero es una variable que almacena la dirección de memoria de otra variable. Se declaran utilizando el operador
\texttt{*} y se accede al valor al que apuntan utilizando el operador \texttt{\&}. Por ejemplo:

\begin{lstlisting}[language=C, caption={Ejemplo de punteros en C}]
#include <stdio.h>

int main() {
  int x = 10; // Declaracion de una variable entera
  int *p = &x; // Declaracion de un puntero que apunta a la direccion de x

  printf("Valor de x: %d\n", x); // Imprime el valor de x
  printf("Direccion de x: %p\n", (void*)&x); // Imprime la direccion de x
  printf("Valor al que apunta p: %d\n", *p); // Imprime el valor al que apunta p (valor de x)
  printf("Direccion almacenada en p: %p\n", (void*)p); // Imprime la direccion almacenada en p

  return 0;
}
\end{lstlisting}


\subsubsection{Funciones}

Las funciones en C++ son bloques de código que realizan una tarea específica y pueden ser reutilizadas en diferentes
partes del programa. Se declaran utilizando la siguiente sintaxis:

\begin{lstlisting}[language=C, caption={Declaración de una función en C}]
int sumar_enteros(int a, int b) {
  return a + b; // Retorna la suma de a y b
}
\end{lstlisting}

Para llamar a una función, simplemente se utiliza su nombre seguido de los argumentos entre paréntesis:

\begin{lstlisting}[language=C, caption={Llamada a una función en C}]
#include <stdio.h>

int sumar_enteros(int a, int b) {
  return a + b;
}

int main() {
  int resultado = sumar_enteros(5, 10); // Llama a la funcion sumar_enteros
  printf("Resultado: %d\n", resultado); // Imprime el resultado
  return 0;
}
\end{lstlisting}

Hay que tener algunas consideraciones cuando se llaman a funciones, porque en C y C++ los argumentos (o parámetros) de
las funciones se pasan \textbf{por valor}, lo que significa que se crea una copia de los argumentos al llamar a la
función. Esto tiene una ventaja y una desventaja. La ventaja es que no se pueden modificar los argumentos originales
desde la función, lo que muchas veces es útil cuando no queremos que la función modifique los datos que le pasamos. La
desventaja es que si los datos que vamos a pasar son muy grandes, la llamada a esa función se vuelve extremádamente
ineficiente. Esto es un problema súper común y hay que tenerlo en cuenta.

En C, la solución a este problema es utilizar punteros para pasar los argumentos que tienen un tamaño grande o que se
desea modificar. Por ejemplo supongamos que queremos pasar un valor que queremos modificar en una función:

\begin{lstlisting}[language=C, caption={Ejemplo de paso por puntero en C}]
#include <stdio.h>

void incrementar(int *x) { // El argumento x se pasa como puntero
  (*x)++; // Incrementa el valor al que apunta x
}

int main() {
  int numero = 5;
  incrementar(&numero); // Llama a la funcion incrementar pasando la direccion de numero
  printf("Numero incrementado: %d\n", numero); // Imprime el numero incrementado
  return 0;
}
\end{lstlisting}

Si se fijan, la forma de llamar a la función incrementar es igual a la sintaxis utilizada en la función \texttt{printf},
porque en ambos casos se está pasando la dirección de memoria de la variable \texttt{numero} para que la función pueda
modificar su valor original.

En C++ se puede hacer lo mismo, pero la sintaxis es mucho más sencilla gracias que tiene una sintaxis especial para
pasar argumentos \textbf{por referencia} que es similar a las direcciones de memoria. La diferencia es que no es
necesario dereferenciarla para poder usarla dentro de la función como sí ocurre dentro de los punteros.

¿Cómo sería la sintaxis para pasar un argumento por referencia en C++? Simplemente se utiliza el operador \texttt{\&}
en la declaración de la función para indicar que el argumento se pasa por referencia, y luego se puede modificar el
valor original desde la función. Por ejemplo:

\begin{lstlisting}[language=C++, caption={Ejemplo de paso por referencia en C++}]
#include <iostream>

using namespace std;

void incrementar(int &x) { // El argumento x se pasa por referencia
  x++; // Incrementa el valor de x
}

int main() {
  int numero = 5;
  incrementar(numero); // Llama a la funcion incrementar
  cout << "Numero incrementado: " << numero << endl; // Imprime el numero incrementado
  return 0;
}
\end{lstlisting}

En definitiva es muy parecido a pasar un puntero en C, con la diferencia de que en C++ no es necesario utilizar el
operador \texttt{*} para declarar un puntero, ni el operador \texttt{\&} para acceder al valor al que apunta el puntero,
sino que se utiliza el operador \texttt{\&} para indicar que el argumento se pasa por referencia, y luego se puede
acceder al valor original directamente desde la función.

\subsubsection{Manejo de memoria}

En C, el manejo de memoria básico de C se realiza utilizando las funciones \texttt{malloc} y \texttt{free} de la
biblioteca \texttt{stdlib.h}. La función \texttt{malloc} se utiliza para asignar memoria dinámica y la función
\texttt{free} se utiliza para liberar la memoria asignada. Por ejemplo:

\begin{lstlisting}[language=C, caption={Manejo de memoria en C}]
#include <stdio.h>
#include <stdlib.h>

int main() {
  int *arr = (int*)malloc(5 * sizeof(int)); // Asigna memoria para un arreglo de 5 enteros
  if (arr == NULL) {
    printf("Error al asignar memoria\n");
    return 1; // Sale del programa si no se pudo asignar memoria
  }

  for (int i = 0; i < 5; i++) {
    arr[i] = i * 10; // Inicializa el arreglo
  }

  for (int i = 0; i < 5; i++) {
    printf("%d ", arr[i]); // Imprime el arreglo
  }
  printf("\n");

  free(arr); // Libera la memoria asignada
  return 0;
}
\end{lstlisting}

Este tipo de cosas pueden resultar un poco complicadas al principio porque ese necesario tener cuidado de no irnos de
los límites de memoria asignada para evitar incurrir en errores de memoria (\textit{segmentation fault}), y también es
necesario recordar liberar la memoria asignada para evitar fugas de memoria (\textit{memory leaks}). Los \textit{memory
leaks} ocurren cuando se asigna memoria dinámica pero no se libera, lo que puede llevar a que el programa consuma cada
vez más memoria y eventualmente se quede sin memoria disponible, lo que puede causar que el programa se bloquee o se
comporte de manera inesperada, a esto se lo conoce muchas veces como OOM (\textit{Out of Memory}).

Sin embargo en C++ tenemos la ventaja de que el manejo de memoria es mucho más sencillo gracias a la utilización de
contenedores de la biblioteca estándar como \texttt{std::vector} que se encargan automáticamente de asignar y liberar
memoria, lo que nos permite centrarnos en la lógica de nuestro programa sin tener que preocuparnos por el manejo de
memoria como veremos en la sección específica de C++ \ref{sec:cpp}.

Sin embargo para algunos ejercicios utilizaremos memoria dinámica en C, por lo cual es importante que entiendan cómo
funciona el manejo de memoria en C, y cómo se puede utilizar para crear estructuras de datos dinámicas como listas
enlazadas, árboles, grafos, etc.

\subsubsection{Matrices}

Al pedir memoria con \texttt{malloc} C nos retorna un puntero a la memoria asignada, y esta memoria asignada es
contigua, con lo cual podemos utilizar punteros o subíndices para acceder a los elementos de la memoria asignada. Por
ejemplo si tenemos un vector de tamaño 100 y queremos acceder al elemento 47 podemos hacerlo de la siguientes maneras:

\begin{lstlisting}[language=C, caption={Acceso a elementos de un vector en C}]
#include <stdio.h>
#include <stdlib.h>

int main() {
  int *vector = (int*)malloc(100 * sizeof(int)); // Asigna memoria para un vector de 100 enteros
  if (vector == NULL) {
    printf("Error al asignar memoria\n");
    return 1; // Sale del programa si no se pudo asignar memoria
  }

  // Acceso al elemento 47 del vector usando subindice
  vector[47] = 42;
  printf("Elemento en la posicion 47: %d\n", vector[47]);

  *(vector + 47) += 1; // Acceso al elemento 47 del vector usando puntero
  printf("Elemento en la posicion 47: %d\n", *(vector + 47));

  free(vector); // Libera la memoria asignada
  return 0;
}
\end{lstlisting}

Supongamos que ahora tenés que trabajar con una matriz de 100x200 enteros (20.000 elementos), en este caso lo que se
puede hacer es calcular la dirección de memoria del elemento en la fila \texttt{fila} y la columna \texttt{col} de la
matriz haciendo lo siguiente:

\begin{lstlisting}[language=C, caption={Acceso a elementos de una matriz en C}]
#include <stdio.h>
#include <stdlib.h>

int main() {
  int filas = 100;
  int columnas = 200;
  int *matriz = (int*)malloc(filas * columnas * sizeof(int)); // Asigna memoria para una matriz de 100x100 enteros
  if (matriz == NULL) {
    printf("Error al asignar memoria\n");
    return 1; // Sale del programa si no se pudo asignar memoria
  }

  int fila = 47;
  int col = 42;

  // Acceso al elemento en la fila y columna especificadas
  int posicion_memoria = fila * columnas + col; // Calcula la posicion en memoria
  matriz[posicion_memoria] = 1234; // Acceso usando subindice
  printf("Elemento en la posicion [%d][%d]: %d\n", fila, col, matriz[posicion_memoria]);

  *(matriz + posicion_memoria) += 1; // Acceso usando puntero
  printf("Elemento en la posicion [%d][%d]: %d\n", fila, col, *(matriz + posicion_memoria));

  free(matriz); // Libera la memoria asignada
  return 0;
}
\end{lstlisting}


\subsubsection{Entrada y salida de datos en C}

C proporciona funciones para la entrada y salida de datos a través de la biblioteca \texttt{stdio.h}. Las funciones más
comunes son:

\begin{itemize}
  \item \texttt{printf}: Para imprimir datos en la consola.
  \item \texttt{scanf}: Para leer datos desde la entrada estándar (teclado).
  \item \texttt{getchar}: Para leer un carácter desde la entrada estándar.
  \item \texttt{putchar}: Para imprimir un carácter en la salida estándar.
  \item \texttt{fgets}: Para leer una línea completa desde la entrada estándar.
  \item \texttt{fputs}: Para imprimir una línea completa en la salida estándar.
\end{itemize}

Estas funciones son ideales para interactuar con el usuario, pero también funcionan para obtener datos de STDIN y
enviarlo a STDOUT, que es lo que se puede utilizar luego para redirigir la entrada y salida de datos desde y hacia
archivos.

Supongamos que tenemos un matriz de \texttt{filas x columnas} enteros y queremos leer los datos desde la entrada
estándar, podemos crear un archivo así:

\begin{verbatim}
3 4
1 2 3 4
5 6 7 8
9 10 11 12
\end{verbatim}

Y luego leerlo desde un programa en C de la siguiente manera:

\begin{lstlisting}[language=C, caption={Lectura de datos desde la entrada estándar en C}]
#include <stdio.h>
#include <stdlib.h>

int main() {
  int filas, columnas;
  scanf("%d %d", &filas, &columnas); // Lee las dimensiones de la matriz

  int **matriz = (int**)malloc(filas * sizeof(int*)); // Asigna memoria para las filas
  for (int i = 0; i < filas; i++) {
    matriz[i] = (int*)malloc(columnas * sizeof(int)); // Asigna memoria para las columnas
  }

  // Lee los datos de la matriz
  for (int i = 0; i < filas; i++) {
    for (int j = 0; j < columnas; j++) {
      scanf("%d", &matriz[i][j]);
    }
  }

  // Imprime la matriz
  printf("Matriz:\n");
  for (int i = 0; i < filas; i++) {
    for (int j = 0; j < columnas; j++) {
      printf("%d ", matriz[i][j]);
    }
    printf("\n");
  }

  return 0;
}
\end{lstlisting}

\subsubsection{Entrada y salida de datos en C++}

En C++, la entrada y salida de datos se realiza a través de la biblioteca \texttt{iostream}, ya que el C++ utiliza
el concepto de flujos para la entrada y salida de datos. Los objetos más comunes para la entrada y salida de datos son:

\begin{itemize}
  \item \texttt{std::cin}: Para leer datos desde la entrada estándar (teclado).
  \item \texttt{std::cout}: Para imprimir datos en la consola.
\end{itemize}

La sintaxis para la entrada y salida de datos en C++ es diferente a la de C, ya que se utilizan los operadores de inserción
(\texttt{<<}) y extracción (\texttt{>>}) para enviar y recibir datos a través de los flujos. Podemos pensar esos
operadores como flechas que indican la dirección del flujo de datos, por ejemplo \texttt{std::cout << "Hola"} indica que
la cadena "Hola" se envía al flujo de salida estándar, mientras que \texttt{std::cin >> x} indica que el valor ingresado
por el usuario se almacena en la variable \texttt{x}.

Como se puede ver no es necesario indicar la dirección de memoria de la variable al leer datos desde la entrada
estándar, como se hace en C con \texttt{scanf}, ya que en C++ el operador de extracción (\texttt{>>}) se encarga de
almacenar el valor ingresado directamente en la variable, lo que hace que la sintaxis sea más sencilla y fácil de
entender.

Por ejemplo si quisiéramos leer una matriz de \texttt{filas x columnas} enteros desde la entrada estándar en C++,
podríamos hacerlo de la siguiente manera:

\begin{lstlisting}[language=C++, caption={Lectura de datos desde la entrada estándar en C++}]
#include <iostream>

using namespace std;

int main() {
  int filas, columnas;
  cin >> filas >> columnas; // lee las dimensiones de la matriz

  int **matriz = new int*[filas]; // asigna memoria para las filas
  for (int i = 0; i < filas; i++) {
    matriz[i] = new int[columnas]; // asigna memoria para las columnas
  }

  // lee los datos de la matriz
  for (int i = 0; i < filas; i++) {
    for (int j = 0; j < columnas; j++) {
      cin >> matriz[i][j];
    }
  }

  // imprime la matriz
  cout << "matriz:" << endl;
  for (int i = 0; i < filas; i++) {
    for (int j = 0; j < columnas; j++) {
      cout << matriz[i][j] << " ";
    }
    cout << endl;
  }
}
\end{lstlisting}

\newpage % ------------------------------------------------------------------------------------------------------------

\subsubsection{Estructuras}

Las estructuras en C son una forma de agrupar diferentes tipos de datos bajo un mismo nombre. Se declaran utilizando la
palabra clave \texttt{struct} y se pueden utilizar para crear tipos de datos personalizados. Por ejemplo:

\begin{lstlisting}[language=C, caption={Declaración de una estructura en C}]
#include <stdio.h>
#include <stdlib.h>

struct Persona {
  char nombre[50]; // Nombre de la persona
  int edad; // Edad de la persona
};

int main() {
  struct Persona* persona = (struct Persona*)malloc(sizeof(struct Persona)); // Asigna memoria para una estructura
  if (persona == NULL) {
    printf("Error al asignar memoria\n");
    return 1; // Sale del programa si no se pudo asignar memoria
  }

  // Inicializa la estructura
  snprintf(persona->nombre, sizeof(persona->nombre), "Juan Perez"); // Asigna el nombre
  persona->edad = 30; // Asigna la edad
  printf("Nombre: %s, Edad: %d\n", persona->nombre, persona->edad); // Imprime los datos de la persona
  free(persona); // Libera la memoria asignada
  return 0;
}
\end{lstlisting}


\newpage % ------------------------------------------------------------------------------------------------------------

\subsubsection{Cadenas en C}

En C, las cadenas de caracteres se representan como arreglos de caracteres terminados en un carácter nulo
(\texttt{NULL}). Es \textbf{muy} importante recordar que para ser una cadena válida, el último carácter debe ser el
carácter nulo, ya que muchas de las funciones de manipulación de cadenas en C dependen de este carácter para determinar
el final de la cadena.

El caracter nulo en C es el caracter con número ASCII 0, y se representa como \texttt{0} o \texttt{NULL}.

Si uno utiliza una cadena en C, el compilador automáticamente agrega el carácter nulo al final de la cadena, por lo que
no es necesario agregarlo. Pero si estamos utilizando memoria dinámica para almacenar una cadena, debemos asegurarnos
nosotros de que el último carácter sea el carácter nulo, para que las funciones de manipulación de cadenas funcionen
bien.

Por ejemplo:

\begin{lstlisting}[language=C, caption={Ejemplo de cadenas en C}]
#include <stdio.h>
#include <string.h> // Incluye la biblioteca para manipulacion de cadenas

int main() {
  char cadena[20]; // Declara un arreglo de 20 caracteres

  cadena[0] = 'H'; // Asigna el primer caracter
  cadena[1] = 'o'; // Asigna el segundo caracter
  cadena[2] = 'l'; // Asigna el tercer caracter
  cadena[3] = 'a'; // Asigna el cuarto caracter

  printf("El largo de cadena es: %d\n", strlen(cadena)); // 'cadena' NO termina en NULO!!!, CUIDADO!!

  char* cadena2 = "Hola"; // Declara una cadena literal, termina en NULO automaticamente
  printf("El largo de cadena2 es: %d\n", strlen(cadena2)); // cadena2 termina en NULO

  return 0;
}
\end{lstlisting}

La salida de este programa (puede) ser:

\begin{verbatim}
El largo de cadena es: 9
El largo de cadena2 es: 4
\end{verbatim}

¿Por qué la cadena \texttt{cadena} tiene un largo de 9? Porque no termina en NULO, y por lo tanto la función
\texttt{strlen} cuenta todos los caracteres hasta que encuentra un caracter \texttt{NULL}. Esto puede llevar a
comportamientos inesperados, ya que si la cadena no termina en NULO, la función \texttt{strlen} seguirá contando hasta
que encuentre un caracter \texttt{NULL} en memoria o hasta que se salga del rango de memoria asignada al programa y
termine con un \texttt{Segmentation Fault}.

Para ello se utiliza la función \texttt{strnlen} que permite especificar el tamaño máximo de la cadena a contar, de esta
forma evitamos tener problemas de memoria, ya que podemos especificar cuántos caracteres vamos a contar como máximo
antes de que la función se salga del rango de memoria asignada al programa.

\begin{lstlisting}[language=C, caption={Ejemplo de cadenas en C con strnlen}]
#include <stdio.h>
#include <string.h> // Incluye la biblioteca para manejo de cadenas

int main() {
  char cadena[20]; // Declara un arreglo de 20 caracteres

  cadena[0] = 'H'; // Asigna el primer caracter
  cadena[1] = 'o'; // Asigna el segundo caracter
  cadena[2] = 'l'; // Asigna el tercer caracter
  cadena[3] = 'a'; // Asigna el cuarto caracter

  printf("El largo de cadena es: %d\n", strnlen(cadena, sizeof(cadena))); // Ahora si termina en NULO!!!

  return 0;
}
\end{lstlisting}

\textbf{Importante}: Esto NO elimina el problema de que la cadena no termine en \texttt{NULL}, para que sólo cuente 4
caracteres deberíamos haber agregado \texttt{cadena[4] = NULL;}, para hacer que nustra cadena \texttt{Hola} termine en
el caracter nulo.

\newpage % ------------------------------------------------------------------------------------------------------------


\subsubsection{Cadenas en C++}

En C++, las cadenas de caracteres se pueden representar utilizando la clase \texttt{string} de la biblioteca
estándar, que proporciona una forma más fácil y segura de manejar cadenas de caracteres en comparación con las cadenas
en C. La clase \texttt{string} se encuentra en la biblioteca \texttt{<string>} y se puede utilizar de la siguiente
manera:

\begin{lstlisting}[language=C++, caption={Ejemplo de cadenas en C++}]
#include <iostream>
#include <string> // Incluye la biblioteca para manejo de cadenas

using namespace std;

int main() {
  string cadena = "Hola"; // Declara una cadena utilizando la clase string
  cout << "La cadena es: " << cadena << endl; // Imprime la cadena
  cout << "El largo de la cadena es: " << cadena.length() << endl; // Imprime el largo de la cadena

  return 0;
}
\end{lstlisting}

Vemos entonces que estas cadenas son mucho más fáciles de manejar que las cadenas en C ya que no hay que hacer ningún
tipo de administración de la memoria ni preocuparse por el carácter nulo al final de la cadena o por el tamaño máximo de
la cadena, ya que la clase \texttt{string} se encarga automáticamente de gestionar la memoria y el tamaño de la cadena.
Además, la clase \texttt{string} proporciona una gran cantidad de funciones para manipular cadenas de caracteres, como
concatenar, comparar, buscar, etc.

\subsection{Scopes}

En C y C++, el \textbf{scope} (o ámbito) de una variable se refiere a la región del código donde la variable es
accesible. Existen diferentes tipos de scopes en C:

\begin{itemize}
  \item \textbf{Scope global}: Las variables declaradas fuera de cualquier función tienen un scope global y son
    accesibles desde cualquier parte del programa.
  \item \textbf{Scope local}: Las variables declaradas dentro de una función tienen un scope local y sólo son
    accesibles dentro de esa función.
  \item \textbf{Scope de bloque}: Las variables declaradas dentro de un bloque (por ejemplo, dentro de llaves
    \texttt{\{\}}) tienen un scope limitado a ese bloque.
\end{itemize}

Saber utilizar los scopes correctamente es fundamental para evitar errores de acceso a variables, por ejemplo supongamos
el siguiente código:

\begin{lstlisting}[language=C, caption={Ejemplo de scopes en C}]
#include <stdio.h>

int x = 10;

void asignar_valor_a_x(int nuevo_x) {
  int x = nuevo_x;
}

int main() {
  int x = 30;
  asignar_valor_a_x(20);
  printf("%d\n", x);
  return 0;
}
\end{lstlisting}

En este caso la salida del programa será \texttt{30}. Y esto se debe a que la variable que se le pasa al \texttt{printf}
es una variable local al \textit{scope} de la función \texttt{main}.

El scope global no tiene ninguna sintaxis particular, simplemente se declara una variable fuera de cualquier función y
en el mismo archivo que estamos trabajando \footnote{En C además existe \texttt{extern} que permite declarar variables
globales en otros archivos, pero no es necesario para esta materia.}. El resto de los scopes se declaran dentro de
llaves \texttt{\{\}}, ya sea scope de funciones o scope de bloques.

Si el scope tiene una sola línea de código, se puede omitir las llaves, excepto en el caso de que se necesite declarar
una función. Ejemplos:

\begin{lstlisting}[language=C, caption={Ejemplo de scopes en C}]
#include <stdio.h>

int x = 10; // scope global

int main() {
  printf("%d\n", x); // imprime el valor de x en el scope global

  int x = 30; // scope local a la funcion main
  printf("%d\n", x); // imprime el valor de x en el scope local de main

  for(int i = 0; i < 5; i++) // scope de bloque
    printf("iteracion %d\n", i); // scope de una linea

  int j = 0;
  while(j < 5) { // scope de bloque
    int cuadrado = j * j; // scope local a la iteracion del while
    printf("cuadrado de %d es %d\n", j, cuadrado);
    j++;
  }

  return 0;
}
\end{lstlisting}

En versiones antiguas de C (C89), las variables había que declararlas al principio del bloque o de las funciones. Sin
embargo eso ya no es necesario en las versiones modernas de C ni en e C++, de hecho mi recomendación para esta materia
es que siempre que sea posible se declare la variable lo más cerca posible de su uso, para evitar confusiones y mejorar
la legibilidad del código. Por ejemplo, en un \texttt{for} es común declarar la variable de iteración dentro del mismo
\texttt{for}.

\newpage

\subsection{Errores}

Hay algo que no se explica en los libros de algoritmos porque excede el ámbito de esos libros, pero los errores de un
lenguaje de programación son un tema fundamental a la hora de programar. Porque no sólo es importante saber escribir
código correcto, sino también es importante saber \textbf{entender} los errores del compilador para poder luego
\textbf{corregirlos}. Es por eso que independientemente del lenguaje de programación que vayan a utilizar en un futuro
profesional, deben conocer los errores y los \textit{warnings} que el compilador (o el intérprete) les entregue para
poder saber qué es lo que está mal.

\subsubsection{Errores de compilación}

Por ejemplo, supongamos el siguiente código:

\begin{lstlisting}[language=C, caption={Código con error de compilación}]
#include <stdio.h>

int main() {
  int n = 10;
  int* ptr = (int*)malloc(n * sizeof(int)); // Error: falta el include <stdlib.h>

  for(int i = 0; i < n; i++) {
    printf("Iteracion %d\n", i);
  }
}
\end{lstlisting}

Al compilar este código, el compilador nos dará un error similar al siguiente:

\begin{verbatim}
ejemplo.c:5:20: error: call to undeclared library function 'malloc'
                with type 'void *(unsigned long)'; ISO C99 and later do not s
upport implicit function declarations [-Wimplicit-function-declaration]
  int* ptr = (int*)malloc(n * sizeof(int)); // Error: falta el include <stdlib.h>
                   ^
ejemplo.c:5:20: note: include the header <stdlib.h> or explicitly provide a declaration for 'malloc'
\end{verbatim}

En este caso, el error nos indica que la función \texttt{malloc} no está declarada, lo que significa que no hemos
incluido la biblioteca \texttt{stdlib.h} que contiene la declaración de la función \texttt{malloc}. Para corregir el
error, simplemente debemos incluir la biblioteca al principio del código:

\begin{lstlisting}[language=C, caption={Código corregido}]
#include <stdio.h>
#include <stdlib.h> // Agregamos la biblioteca stdlib.h

int main() {
  int n = 10;
  int* ptr = (int*)malloc(n * sizeof(int)); // Ahora no hay error

  for(int i = 0; i < n; i++) {
    printf("Iteracion %d\n", i);
  }

  free(ptr); // No olvides liberar la memoria asignada
}
\end{lstlisting}

A esto se le llama \textbf{error de compilación}. Son los errores más fáciles de detectar porque el compilador nos
informa de ellos antes de ejecutar el programa, sólo tenemos que ser cuidadosos sobre el mensaje de error que nos
entrega el compilador.

\subsubsection{Errores de ejecución}

Los errores de ejecución son aquellos que ocurren mientras el programa está ejecutando, con lo cual son errores mucho
más difíciles de detectar, porque pueden ocurrir en cualquier momento y no siempre son evidentes. Por ejemplo, si
tenemos el siguiente código:

\begin{lstlisting}[language=C, caption={Código con error de ejecución}]
#include <stdio.h>
#include <stdlib.h>

int main() {
  int m = 10;
  int n = 10000;
  int* ptr = (int*)malloc(n * m * sizeof(int));

  for(int i = 0; i < n*n; i++) { // deberia ser n * m
    ptr[i] = i;
  }

  free(ptr);
}
\end{lstlisting}

Al ejecutar este código, el programa puede fallar con un error de segmentación (segmentation fault) porque estamos
intentando acceder a una posición de memoria que no hemos asignado. El error de segmentación ocurre cuando el programa
intenta acceder a una dirección de memoria que no le corresponde y el sistema operativo lo detecta, terminando el
programa abruptamente.

\begin{verbatim}
$ ./ejemplo 
[1]    29897 segmentation fault  ./ejemplo
\end{verbatim}

\subsection{C++}
\label{sec:cpp}

Hasta acá lo que vimos fueron los conceptos básicos de C y algunas cosas que son paralelas en C++. Si bien C++ se lo
asocia con la programación orientada a objetos (OOP), en esta materia no vamos a ver nada de OOP, pero sí vamos a
utilizar los contenedores de C++ que son parate de la biblioteca estándar de C++ (STL) y que nos permiten manejar
estructuras de datos como vectores, listas, colas, pilas, etc. de una forma mucho más sencilla que en C, ya que se
encargan automáticamente de gestionar la memoria y el tamaño de las estructuras de datos, lo que nos permite centrarnos
en la lógica de nuestro programa sin tener que preocuparnos por el manejo de memoria como vimos anteriormente.

\subsubsection{Contenedores de la biblioteca estándar de C++}

Los contenedores de la biblioteca estándar de C++ (STL) son una colección de clases y funciones que proporcionan
estructuras de datos que manejan la memoria por nosotros y tienen una interfaz fácil de usar. Algunos de los
contenedores más comunes son:

\begin{itemize}
  \item \texttt{std::vector}: Un arreglo dinámico que puede cambiar de tamaño automáticamente.
  \item \texttt{std::deque}: Una cola de doble extremo que permite inserciones y eliminaciones eficientes en ambos extremos.
  \item \texttt{std::set}: Un conjunto que almacena elementos únicos en orden.
  \item \texttt{std::unordered\_set}: Un conjunto que almacena elementos únicos sin orden.
  \item \texttt{std::map}: Un mapa que almacena pares clave-valor en orden.
  \item \texttt{std::unordered\_map}: Un mapa que almacena pares clave-valor sin orden.
\end{itemize}

Cuando uno utiliza estos contenedores, no es necesario preocuparse por el manejo de memoria, ya que se encargan
automáticamente de asignar y liberar memoria según sea necesario. Además, estos contenedores proporcionan una gran
cantidad de funciones para manipular los datos almacenados, como insertar, eliminar, encontrar el primer elemento, el
último elemento.

\subsubsection{Iteradores}

Intrínsecamente relacionados con los contenedores de la biblioteca estándar de C++ (STL) están los iteradores, que son
objetos que permiten recorrer los elementos de un contenedor de manera secuencial. Los iteradores proporcionan una
interfaz uniforme para acceder a los elementos de diferentes tipos de contenedores, lo que facilita la manipulación de
datos sin tener que preocuparse por la estructura interna del contenedor. Por ejemplo, si queremos recorrer un vector de
enteros utilizando un iterador, podríamos hacerlo de la siguiente manera:

\begin{lstlisting}[language=C++, caption={Ejemplo de iteradores en C++}]
#include <iostream>
#include <vector>

using namespace std;

int main() {
  vector<int> numeros = {1, 2, 3, 4, 5}; // Crea un vector de enteros

  // Crea un iterador para el vector
  vector<int>::iterator it;

  // Recorre el vector utilizando el iterador
  for (it = numeros.begin(); it != numeros.end(); ++it) {
    cout << *it << " "; // Imprime el valor al que apunta el iterador
  }
  cout << endl;

  return 0;
}
\end{lstlisting}

Expliquemos un poco lo que está pasando acá. Primero, creamos un vector de enteros llamado \texttt{numeros} e
inicializamos con algunos valores.

Vemos que \texttt{numeros.begin()} y \texttt{numeros.end()} son funciones que devuelven iteradores que apuntan al primer
elemento del vector y al elemento siguiente al último elemento del vector. Con lo cual \texttt{numeros.begin()} es un
iterador que apunta al primer elemento del vector (y de cualquier otro contenedor), mientras que \texttt{numeros.end()}
es un iterador que apunta a una posición de memoria que está justo después del último elemento del vector, lo que
significa que no es un iterador válido para acceder a los elementos del vector, pero es útil para determinar cuándo
hemos llegado al final del vector durante la iteración. A esto en matemáticas se lo conoce como que un conjunto está
cerrado por abajo y abierto por arriba, porque \texttt{numeros.begin()} pertenece al conjunto pero
\texttt{numeros.end()} no pertenece al conjunto.

Lo interesante de este tipo de iteradores es que se pueden utilizar con cualquier tipo de contenedor de la biblioteca
estándar de C++, por ejemplo imaginemos que hacemos lo mismo pero con un \texttt{std::set}:

\begin{lstlisting}[language=C++, caption={Ejemplo de iteradores con un set en C++}]
#include <iostream>
#include <set>

using namespace std;

int main() {
  set<int> numeros; // Crea un set de enteros
  numeros.insert(5);
  numeros.insert(3);
  numeros.insert(1);
  numeros.insert(4);
  numeros.insert(2);

  // Crea un iterador para el set
  set<int>::iterator it;

  // Recorre el set utilizando el iterador
  for (it = numeros.begin(); it != numeros.end(); ++it) {
    cout << *it << " "; // Imprime el valor al que apunta el iterador
  }
  cout << endl;

  return 0;
}
\end{lstlisting}

De todas formas C++ también permite el uso de \texttt{auto} para declarar iteradores de forma automática, lo que hace
que el código sea más limpio y fácil de leer, por ejemplo:

\begin{lstlisting}[language=C++, caption={Ejemplo de iteradores con auto en C++}]
#include <iostream>
#include <set>

using namespace std;

int main() {
  set<int> numeros = {5, 3, 1, 4, 2}; // Crea un set de enteros

  // Recorre el set utilizando un iterador declarado con auto
  for (auto e : numeros) {
    cout << e << " "; // Imprime el valor del elemento
  }
  cout << endl;
  return 0;
}
\end{lstlisting}

Pero para esto hay que utilizar las extesiones de C++11, que es una versión de C++ que introdujo muchas mejoras y nuevas
características al lenguaje, incluyendo el uso de \texttt{auto} para la declaración automática de variables, lo que hace
que el código sea más limpio y fácil de leer, ya que no es necesario especificar el tipo de la variable cuando se
declara, sino que el compilador lo infiere automáticamente a partir del valor que se le asigna. Otro cambio es que se
puede inicializar los \texttt{std::set} utilizando una lista \texttt{\{5, 3, 1, 4, 2\}} en lugar de tener que insertar
cada elemento por separado utilizando la función \texttt{insert}.

Para poder compilar hay que explicitar el uso de C++11 utilizando la opción \texttt{-std=c++11} al compilar, por
ejemplo:

\begin{verbatim}
g++ -std=c++11 ejemplo.cpp -o ejemplo
\end{verbatim}

\input{../shared.tex/common_footers.tex}

\end{document}
